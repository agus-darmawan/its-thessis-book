\begin{center}
  \large\textbf{ABSTRACT}
\end{center}

\addcontentsline{toc}{chapter}{ABSTRACT}

\vspace{2ex}

\begingroup
% Menghilangkan padding
\setlength{\tabcolsep}{0pt}

\noindent
\begin{tabularx}{\textwidth}{l >{\centering}m{3em} X}
  \emph{Name}     & : & \name{}         \\

  \emph{Title}    & : & \engtatitle{}   \\

  \emph{Advisors} & : & 1. \advisor{}   \\
                  &   & 2. \coadvisor{} \\
\end{tabularx}
\endgroup

% Abstract content
The increase in electricity consumption in Indonesia drives the development of electrical infrastructure, particularly substations, which are prone to overheating issues. This problem can cause severe damage and disrupt electricity distribution. This research proposes an overheating monitoring system based on an autonomous quadruped legged robot equipped with a thermal camera and the YOLOv8 object detection model. The robot is designed to detect abnormal temperatures on substation components in real-time, using a localization and mapping system that combines efficient and accurate mapping methods. This integration allows the robot to navigate along patrol routes using PID and Pure Pursuit controllers, as well as obstacle avoidance based on Braitenberg, which is conditionally activated when objects are detected within a certain distance. This approach allows the robot to continue following the pre-recorded path while dynamically reacting to obstacles without having to re-plan the entire route. Such integration can be realized through a behavior-based arbitration or subsumption architecture scheme, where the obstacle avoidance module takes higher priority when critical conditions are detected. Once the environment is safe again, control will be shifted back to the main path controller. Testing results show that the robot is capable of performing mapping and localization with high accuracy, as well as successfully avoiding obstacles in a dynamic environment with a 95\% success rate. The system also effectively detects components experiencing overheating with an adequate detection accuracy. The system is equipped with a web interface built with React, making it easier for operators to monitor and control the robot remotely. The results of this research are expected to contribute to the development of robotic technology for more efficient and reliable monitoring of electrical infrastructure.

\vspace{2ex}
\noindent
\textbf{Keywords:} \emph{Overheat}, \emph{Quadruped robot}, \emph{thermal camera}, YOLOv8, Fast LIO