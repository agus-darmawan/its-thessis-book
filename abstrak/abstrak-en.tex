\begin{center}
  \large\textbf{ABSTRACT}
\end{center}

\addcontentsline{toc}{chapter}{ABSTRACT}

\vspace{2ex}

\begingroup
\setlength{\tabcolsep}{0pt}
\noindent
\begin{tabularx}{\textwidth}{l >{\centering}m{3em} X}
  \emph{Name}     & : & \name{}         \\
  \emph{Title}    & : & \engtatitle{}   \\
  \emph{Advisors} & : & 1. \advisor{}   \\
                  &   & 2. \coadvisor{} \\
\end{tabularx}
\endgroup

\vspace{2ex}

The increasing demand for electrical energy in Indonesia requires a more reliable and intelligent approach to maintaining substation infrastructure. Overheating in high-voltage substation components can cause equipment failure and service disruption, making real-time monitoring essential. This research develops a quadruped-legged robot system designed to detect overheating components in electrical substations using a thermal camera and onboard intelligence. The system integrates a YOLOv8-based object detection model to identify electrical components and performs temperature analysis within the detected regions of interest to determine possible overheat conditions. The navigation system combines PID control with the Pure Pursuit algorithm, allowing the robot to follow predefined patrol routes while recording thermal data at inspection points. Mapping is performed using the FastLIO2 SLAM algorithm, producing a 3D point cloud of the environment. However, due to the lack of visual diversity in substation structures, the localization process relies on odometry with fixed initial positioning. A control station is implemented to monitor and receive real-time data via WebSocket and WebRTC, including position, sensor status, and thermal images. The system achieved detection accuracy with a mean average precision (mAP) of 0.8554 and a precision of 0.9529 for two classes of components. Navigation tests yielded average position errors ranging from 0.40 to 0.85 meters depending on the PID parameters and lookahead values. Data transmission remained stable with average delays of 30--50 ms for telemetry and 210--350 ms for video streaming over distances of up to 100 meters. Simulation testing using ROS bag data demonstrated the system's ability to detect overheating conditions and predict component positions, with data successfully transmitted to the web-based interface. This study demonstrates the feasibility of implementing a mobile quadruped monitoring system for autonomous inspection in substation environments, offering potential for safer and more efficient thermal diagnostics.

\vspace{2ex}
\noindent
\textbf{Keywords:} \emph{Overheat}, \emph{quadruped robot}, \emph{thermal camera}, YOLOv8, FastLIO2
