\begin{center}
  \large\textbf{ABSTRAK}
\end{center}

\addcontentsline{toc}{chapter}{ABSTRAK}

\vspace{2ex}

\begingroup
% Menghilangkan padding
\setlength{\tabcolsep}{0pt}

\noindent
\begin{tabularx}{\textwidth}{l >{\centering}m{2em} X}
  Nama Mahasiswa    & : & \name{}         \\

  Judul Tugas Akhir & : & \tatitle{}      \\

  Pembimbing        & : & 1. \advisor{}   \\
                    &   & 2. \coadvisor{} \\
\end{tabularx}
\endgroup

% Isi Abstrak
Peningkatan konsumsi listrik di Indonesia mendorong pengembangan infrastruktur kelistrikan, terutama gardu listrik yang rentan terhadap masalah \emph{overheating}. Masalah ini dapat menyebabkan kerusakan serius dan gangguan distribusi listrik. Penelitian ini mengusulkan sistem pemantauan \emph{overheat} berbasis \emph{autonomous quadruped legged robot} yang dilengkapi dengan \emph{thermal camera} dan model deteksi objek \emph{YOLOv8}. Robot ini dirancang untuk mendeteksi suhu abnormal pada komponen gardu listrik secara \emph{real-time}, menggunakan sistem lokalisasi dan pemetaan yang menggabungkan metode pemetaan yang efisien dan akurat. Integrasi ini memungkinkan robot untuk melakukan navigasi dengan mengikuti jalur patroli menggunakan pengendali \emph{PID} dan \emph{Pure Pursuit}, serta penghindaran rintangan berbasis \emph{Braitenberg} yang diaktifkan secara kondisional saat objek terdeteksi dalam jarak tertentu. Pendekatan ini memungkinkan robot untuk tetap mengikuti jalur yang telah direkam, namun tetap mampu bereaksi secara dinamis terhadap rintangan tanpa harus merencanakan ulang lintasan secara menyeluruh. Integrasi semacam ini dapat direalisasikan melalui skema \emph{behavior-based arbitration} atau \emph{subsumption architecture}, di mana modul penghindaran rintangan memiliki prioritas lebih tinggi ketika kondisi kritis terdeteksi. Setelah lingkungan kembali aman, kontrol akan dialihkan kembali ke pengendali lintasan utama. Hasil pengujian menunjukkan bahwa robot mampu melakukan \emph{mapping} dan \emph{localization} dengan akurasi tinggi, serta berhasil menghindari rintangan dalam lingkungan yang dinamis dengan tingkat keberhasilan 95\%. Sistem juga berhasil mendeteksi komponen yang mengalami \emph{overheat} secara efektif dengan tingkat akurasi deteksi yang memadai. Sistem dilengkapi dengan antarmuka \emph{web} yang dibangun dengan React, memudahkan operator untuk memantau dan mengontrol robot dari jarak jauh. Hasil penelitian ini diharapkan berkontribusi pada pengembangan teknologi robotika untuk pemantauan infrastruktur kelistrikan yang lebih efisien dan andal.

\vspace{2ex}
\noindent
\textbf{Kata Kunci:} \emph{Overheat}, \emph{Quadruped robot}, \emph{thermal camera}, YOLOv8, Fast LIO