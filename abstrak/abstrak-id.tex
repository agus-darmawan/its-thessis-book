\begin{center}
  \large\textbf{ABSTRAK}
\end{center}

\addcontentsline{toc}{chapter}{ABSTRAK}

\vspace{2ex}

\begingroup
\setlength{\tabcolsep}{0pt}
\noindent
\begin{tabularx}{\textwidth}{l >{\centering}m{2em} X}
  Nama Mahasiswa    & : & \name{}         \\
  Judul Tugas Akhir & : & \tatitle{}      \\
  Pembimbing        & : & 1. \advisor{}   \\
                    &   & 2. \coadvisor{} \\
\end{tabularx}
\endgroup

\vspace{2ex}
  Peningkatan populasi dan pertumbuhan industri yang pesat menyebabkan permintaan energi listrik di Indonesia terus mengalami lonjakan signifikan setiap tahunnya. Kondisi ini menuntut penguatan infrastruktur kelistrikan, terutama pada gardu induk yang berperan penting dalam mendistribusikan dan mengendalikan aliran listrik. Salah satu permasalahan yang sering terjadi di gardu induk adalah \emph{overheat} pada komponen kritis seperti transformator, isolator, dan \emph{disconnector}, yang dapat mengganggu stabilitas sistem dan menimbulkan risiko kerusakan serius.  Penelitian ini mengembangkan sistem robot berkaki empat (\emph{quadruped-legged robot}) yang dirancang untuk mendeteksi komponen gardu induk yang mengalami \emph{overheat}. Sistem ini mengintegrasikan model deteksi objek berbasis \emph{YOLOv8} untuk mengenali komponen listrik, serta menganalisis suhu pada area hasil deteksi guna mengidentifikasi potensi kondisi \emph{overheat}. Navigasi robot dikendalikan menggunakan kombinasi algoritma \emph{PID} dan \emph{Pure Pursuit} agar mampu mengikuti lintasan patroli secara otonom, sementara proses pemetaan dan lokalisasi dilakukan secara simultan menggunakan algoritma \emph{FastLIO2} untuk menghasilkan peta awan titik tiga dimensi dan estimasi posisi yang akurat. Sistem \emph{control station} dikembangkan untuk menerima dan memvisualisasikan data posisi, status sensor, serta citra termal secara waktu nyata menggunakan protokol \emph{WebSocket} dan \emph{WebRTC}. Hasil pengujian menunjukkan bahwa sistem berhasil mendeteksi \emph{overheat} dengan akurasi tinggi, ditandai dengan nilai \emph{mean average precision (mAP)} sebesar 0{,}8554 dan presisi sebesar 0{,}9529 untuk dua kelas komponen. Pengujian navigasi menunjukkan galat posisi rata-rata antara 0{,}40 hingga 0{,}85 meter, bergantung pada parameter \emph{PID} dan nilai \emph{lookahead}. Transmisi data berjalan stabil dengan rata-rata \emph{delay} 30--50 ms untuk data status dan 210--350 ms untuk \emph{video streaming} dalam jarak hingga 100 meter. Pengujian simulasi menggunakan data \emph{rosbag} juga membuktikan kemampuan sistem dalam mendeteksi \emph{overheat} dan mengirimkan data ke antarmuka web secara berhasil. Hasil ini menunjukkan bahwa sistem pemantauan bergerak berbasis robot \emph{quadruped} layak diimplementasikan sebagai solusi inspeksi otonom pada gardu induk, dengan potensi besar dalam meningkatkan efisiensi dan keamanan infrastruktur kelistrikan nasional.
  

\vspace{2ex}
\noindent
\textbf{Kata Kunci:} \emph{Overheat}, \emph{robot quadruped}, \emph{kamera termal}, YOLOv8, FastLIO2
