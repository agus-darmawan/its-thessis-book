\begin{center}
  \Large
  \textbf{KATA PENGANTAR}
\end{center}

\addcontentsline{toc}{chapter}{KATA PENGANTAR}

\vspace{2ex}

% Ubah paragraf-paragraf berikut dengan isi dari kata pengantar

Puji syukur kami panjatkan kehadirat Ida Sang Hyang Widhi Wasa  sehingga penelitian ini dapat disusun dan diselesaikan dengan baik. Dalam proses penyusunan tugas akhir ini, banyak pihak yang telah memberikan dukungan, bimbingan, dan motivasi kepada penulis. Oleh karena itu, dengan segala hormat dan rasa terima kasih  penulis menyampaikan penghargaan kepada:

\begin{enumerate}[nolistsep]
  \item Keluarga tercinta, khususnya Ibu Ni Nengah Sudini, Bapak I Nyoman Subina, serta saudara yang selalu memberikan doa dan dukungan, baik secara moral maupun material.
  \item Kepala Departemen \studyprogram{}, Bapak \headofdepartment{}, yang telah memberikan dukungan dalam kelancaran studi serta penyusunan tugas akhir ini.
  \item Bapak \advisor{} dan Bapak \coadvisor{}, selaku dosen pembimbing yang telah memberikan bimbingan, arahan, serta motivasi kepada penulis dalam menyelesaikan tugas akhir ini.
  \item Bapak Dr. Rudy Dikairono, S.T., M.T., selaku dosen pembimbing dari Tim Barunastra ITS yang telah memberikan dukungan, serta motivasi kepada penulis dalam menyelesaikan studi dan tugas akhir ini.
  \item Ezra Robotics dan KIIK Robotics yang telah memberikan dukungan berupa perangkat keras untuk keperluan penelitian tugas akhir ini.
  \item Keluarga besar Laboratorium B401, atas kebersamaan, bantuan, serta dukungan selama proses penelitian ini.
  \item Tim Barunastra ITS, atas pengalaman, kerja sama, dan semangat yang diberikan selama perjalanan akademik ini.
  \item Teman-teman seperjuangan: Gustu Dharma, Deva Febriana, Agus Ginting, Zein Bachtiar, Rifki Qolby, Jeremy Jhonson, Vibra Dananjaya, Clive Kosasih, dan Ikhsan Moekhtar, yang selalu memberikan dukungan dan semangat dalam perjalanan akademik ini.
  \item Seorang wanita yang menjadi sumber motivasi dan menjadikan perjalanan penyusunan tugas akhir ini lebih bermakna.
\end{enumerate}

Penulis menyadari bahwa penelitian ini masih jauh dari sempurna. Oleh karena itu, segala kritik dan saran yang membangun akan diterima dengan penuh keterbukaan demi perbaikan ke depannya.  Demikian, semoga penelitian ini dapat bermanfaat bagi perkembangan ilmu pengetahuan serta menjadi inspirasi bagi para pembaca.

\begin{flushright}
  \begin{tabular}[b]{c}
    \place{}, \MONTH{} \the\year{} \\
    \\
    \\
    \\
    \\
    \name{}
  \end{tabular}
\end{flushright}
