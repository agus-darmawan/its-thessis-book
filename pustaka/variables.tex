% Atur variabel berikut sesuai namanya

% nama
\newcommand{\name}{I Wayan Agus Darmawan}
\newcommand{\authorname}{Darmawan, Agus}
\newcommand{\nickname}{Darmawan}
\newcommand{\advisor}{Prof. Dr. Ir. Mauridhi Hery Purnomo, M.Eng.}
\newcommand{\coadvisor}{Muhtadin, S.T., M.T.}
\newcommand{\examinerone}{Dr. Ahmad Zaini, S.T., M.T.}
\newcommand{\examinertwo}{Eko Pramunanto, S.T., M.T.}
\newcommand{\examinerthree}{Ir. Hanny Boedinoegroho, M.T.}
\newcommand{\headofdepartment}{Dr. Arief Kurniawan, S.T., M.T.}

% identitas
\newcommand{\nrp}{5024 21 1070}
\newcommand{\advisornip}{19580916198601 1 001}
\newcommand{\coadvisornip}{19810609200912 1 003}
\newcommand{\examineronenip}{19750419200212 1 003}
\newcommand{\examinertwonip}{19661203199412 1 001}
\newcommand{\examinerthreenip}{19610706198701 1 001}
\newcommand{\headofdepartmentnip}{19740907200212 1 001}

% judul
\newcommand{\tatitle}{PENGEMBANGAN ROBOT \emph{QUADRUPED-LEGGED} UNTUK ESTIMASI POSISI KOMPONEN \emph{OVERHEAT} PADA GARDU INDUK BERBASIS KAMERA TERMAL}
\newcommand{\engtatitle}{DEVELOPMENT OF A QUADRUPED-LEGGED ROBOT FOR ESTIMATING THE POSITION OF OVERHEATED COMPONENTS IN ELECTRICAL SUBSTATIONS USING A THERMAL CAMERA}

% tempat
\newcommand{\place}{Surabaya}

% jurusan
\newcommand{\studyprogram}{Teknik Komputer}
\newcommand{\engstudyprogram}{Computer Engineering}

% fakultas
\newcommand{\faculty}{Teknologi Elektro dan Informatika Cerdas}
\newcommand{\engfaculty}{Intelligent Electrical and Informatics Technology}

% singkatan fakultas
\newcommand{\facultyshort}{FT-EIC}
\newcommand{\engfacultyshort}{F-ELECTICS}

% departemen
\newcommand{\department}{Teknik Komputer}
\newcommand{\engdepartment}{Computer Engineering}

% kode mata kuliah
\newcommand{\coursecode}{EC184801}