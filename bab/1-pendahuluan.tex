\chapter{PENDAHULUAN}
\label{chap:pendahuluan}

\section{Latar Belakang}
\label{sec:latarbelakang}
\sloppy
Sektor kelistrikan di Indonesia memiliki peranan yang sangat penting dalam mendukung kehidupan masyarakat. Populasi penduduk yang terus meningkat dan perkembangan sktor industri yang pesat, menyebabkan kebutuhan listrik mengalami lonjakan yang signifikan. Menurut Laporan Statistik PLN tahun 2024, konsumsi listrik di Indonesia pada tahun 2024 mencapai 306.219,42 GWh dengan jumlah pelanggan sebanyak 92.877.292. Angka ini mencerminkan tren peningkatan yang konsisten dibandingkan tahun-tahun sebelumnya. Sebagai perbandingan, pada tahun 2023 konsumsi listrik tercatat sebesar 288.435,78 GWh dengan jumlah pelanggan sebanyak 89.153.278, sedangkan pada tahun 2022 mencapai 273.761,48 GWh dengan 85.636.198 pelanggan \cite{PLN2024}. Peningkatan konsumsi listrik yang signifikan memerlukan penguatan infrastruktur distribusi tenaga listrik untuk  memenuhi permintaan. Salah satu elemen infrastruktur yang sangat vital dalam sistem distribusi tenaga listrik adalah gardu induk. gardu induk berfungsi sebagai penghubung antara pembangkit listrik dan konsumen. Gardu tidak hanya berfungsi mendistribusikan tenaga listrik, tetapi juga mengatur dan mengontrol aliran listrik, menjaga kualitas daya, serta melindungi sistem dari gangguan.


Salah satu masalah yang sering terjadi pada gardu induk adalah \emph{overheat} pada komponen kritis di dalamnya. \emph{Overheat} terjadi ketika suhu komponen melebihi batas aman, yang dapat disebabkan oleh berbagai faktor, termasuk beban berlebih, kondisi lingkungan yang ekstrem, dan kurangnya pemeliharaan yang tepat\cite{Bailey2022}. Ketika suhu komponen meningkat, risiko kerusakan menjadi lebih tinggi, yang dapat mengakibatkan penurunan efisiensi operasional dan potensi pemadaman listrik yang tidak diinginkan\cite{Aksenovich2022}. Studi kasus menunjukkan bahwa \emph{overheat} pada transformator dapat menyebabkan kerusakan serius pada isolasi minyak, yang berfungsi untuk mendinginkan dan melindungi komponen internal dari arus listrik. Ketika suhu minyak isolasi meningkat, sifat dielektriknya dapat terdegradasi, yang berpotensi menyebabkan kegagalan isolasi dan kebakaran\cite{Kalathiripi2017}. Selain itu, \emph{overheat} juga dapat memengaruhi komponen lain dalam gardu induk, seperti isolator dan \emph{disconnector}. Ketika isolator mengalami \emph{overheat}, material isolasi dapat terdegradasi, sehingga kemampuan menahan tegangan menurun dan meningkatkan risiko terjadinya \emph{arcing} atau percikan listrik\cite{Li2017}.

Mencegah \emph{overheat} pada komponen gardu induk, sangat penting untuk melakukan pemeliharaan yang tepat dan pemantauan suhu secara berkala. Dalam konteks ini, penerapan solusi otomatisasi berbasis robotika dapat menjadi alternatif yang efektif untuk meningkatkan efektivitas pemantauan dan pemeliharaan. Salah satu teknologi yang relevan adalah \emph{autonomous mobile robot}, yaitu perangkat yang dirancang untuk melaksanakan berbagai tugas secara otomatis, termasuk tugas yang membutuhkan ketelitian tinggi atau pengawasan di area yang sulit dijangkau manusia. Salah satu jenis robot yang saat ini sedang berkembang pesat adalah robot \emph{quadruped-legged}. Robot jenis ini dilengkapi dengan empat kaki untuk bergerak dengan meniru cara gerak hewan seperti anjing atau kuda. \emph{Quadruped-legged} memiliki keunggulan dibandingkan robot beroda, khususnya dalam hal mobilitas dan kemampuan bermanuver di medan yang tidak rata. Beberapa riset terkait \emph{quadruped-legged robot} untuk pemantauan gardu induk telah dilakukan, seperti pada proyek \emph{ASUMO (Advanced Substation Monitoring)}. Proyek ini menunjukkan bahwa penggunaan robot berkaki empat dapat menjadi alternatif yang efektif untuk meningkatkan efisiensi operasional dan menjaga kestabilan pasokan listrik \cite{ASUMO2023}.  

Robot \emph{quadruped-legged}  dapat dilengkapi dengan \emph{thermal camera} untuk mendeteksi suhu pada komponen gardu induk, sehingga memungkinkan deteksi dini terhadap \emph{overheat} dan pencegahan kerusakan yang lebih serius. Selain itu, integrasi teknologi robotika dengan sistem informasi berbasis \emph{Internet of Things (IoT)} memungkinkan pemantauan \emph{real-time} dan analisis data yang lebih akurat dan cepat. Dengan pendekatan ini, operator dapat secara proaktif mengambil tindakan preventif untuk mengatasi permasalahan teknis yang muncul, meningkatkan respons terhadap gangguan, dan mengurangi waktu henti operasional. Oleh karena itu, penelitian lebih lanjut dan pengembangan teknologi robotika dalam konteks pemantauan infrastruktur kelistrikan sangat diperlukan untuk memastikan sistem kelistrikan Indonesia dapat berfungsi secara optimal dan berkelanjutan di masa depan serta mendukung automatisasi dalam pemantauan dan pemeliharaan gardu induk dengan menggunakan teknologi robotika.

\section{Permasalahan}

Penelitian ini berfokus pada pengembangan robot otonom berkaki empat untuk mengestimasi posisi komponen gardu induk yang berpotensi mengalami \emph{overheat} berdasarkan citra dari kamera termal. Dalam pengembangannya, terdapat beberapa tantangan utama. Pertama, dibutuhkan sistem \emph{computer vision} untuk mendeteksi jenis dan posisi komponen pada citra termal serta mengidentifikasi indikasi \emph{overheat} berdasarkan distribusi suhu. Kedua, robot harus mampu melakukan navigasi secara otonom di lingkungan gardu induk yang kompleks dan penuh rintangan untuk mencapai sudut pandang optimal terhadap komponen yang dipantau. Ketiga, sistem perlu dilengkapi \emph{control station} yang dapat memantau pergerakan robot dan menampilkan hasil analisis suhu serta estimasi posisi komponen yang terdeteksi mengalami \emph{overheat} secara \emph{real-time}.

\vspace{2ex}


\section{Tujuan}
\label{sec:Tujuan}

Penelitian ini bertujuan mengembangkan sistem robotik otonom berbasis kamera termal untuk estimasi posisi komponen gardu induk yang berpotensi \emph{overheat}. Tujuan khusus meliputi:

\begin{enumerate}
      \item Mengembangkan sistem \emph{computer vision} yang mampu mendeteksi dan mengenali komponen gardu induk serta mengidentifikasi indikasi \emph{overheat} berdasarkan analisis citra termal, sebagai dasar estimasi posisi komponen yang berpotensi \emph{overheat}.
    
      \item Merancang dan mengimplementasikan sistem navigasi otonom yang memungkinkan robot menjelajahi lingkungan gardu induk secara mandiri, menghindari rintangan, dan mencapai sudut pandang optimal untuk pengambilan data termal dari setiap komponen gardu induk.
    
      \item Membangun \emph{control station} yang berfungsi untuk memantau pergerakan robot secara \emph{real-time}, menampilkan hasil analisis suhu, serta memberikan estimasi posisi dari komponen yang terdeteksi mengalami potensi \emph{overheat}.
    \end{enumerate}

    
  \newpage  
\section{Batasan Masalah}
Penelitian ini memiliki sejumlah batasan yang ditetapkan untuk menyesuaikan dengan keterbatasan sumber daya, waktu, dan ruang lingkup. Adapun batasan-batasan tersebut adalah sebagai berikut:

\begin{enumerate}
    \item\textbf{Perangkat yang digunakan} \\
    Penelitian ini dilakukan menggunakan robot DeepRobotics X30 Pro versi 0.0.8-0 (2024) yang dilengkapi dengan \emph{ PTZ thermal camera} Hikmicro HM-TD5528T, serta \emph{AP router} Doublecom DB6000ANLT90-FR (\emph{base station}) dan Doublecom DB6000FR-ANS (\emph{rover}) sebagai perangkat komunikasi. Seluruh eksperimen dan pengambilan data dibatasi pada konfigurasi dan karakteristik perangkat ini.
    
    \item \textbf{Algoritma pergerakan robot} \\
    Sistem pergerakan robot menggunakan \emph{algoritma motion} bawaan pabrik, sehingga penelitian ini tidak membahas aspek kinematika, dinamika, maupun stabilitas robot di berbagai jenis medan.

    \item \textbf{Ruang lingkup pengembangan sistem} \\
    Pengembangan sistem dibatasi pada otomatisasi proses pemantauan komponen gardu induk yang sebelumnya dilakukan secara manual. Penelitian ini tidak mencakup pengembangan skenario pemantauan baru.

    \item \textbf{Lingkungan pengujian} \\
    Pengujian dilakukan di ingkungan nyata dan simulasi. Simulasi dilakukan menggunakan data \emph{log} hasil pengujian langsung. Sehingga penelitian tidak membangun lingkungan simulasi baru.
\end{enumerate}




\section{Sistematika Penulisan}
\label{sec:sistematikapenulisan}

Laporan penelitian tugas akhir ini disusun dalam lima bab, yaitu sebagai berikut:

\begin{enumerate}[nolistsep]
  \item \textbf{BAB I Pendahuluan}

        Bab ini menjelaskan latar belakang penelitian, rumusan masalah, tujuan penelitian, batasan masalah, serta sistematika penulisan laporan tugas akhir ini.

        \vspace{2ex}

  \item \textbf{BAB II Tinjauan Pustaka}

        Bab ini menguraikan teori-teori dan konsep-konsep yang relevan serta penelitian terdahulu yang mendukung dan menjadi landasan dalam penelitian ini.

        \vspace{2ex}

  \item \textbf{BAB III Desain dan Implementasi Sistem}
        Bab ini membahas mengenai perancangan sistem yang dilakukan, serta implementasi teknis dari sistem yang diterapkan dalam penelitian ini.

        \vspace{2ex}

  \item \textbf{BAB IV Pengujian dan Analisis}

        Bab ini menyajikan hasil pengujian sistem yang telah diimplementasikan serta analisis terhadap kinerja dan efektivitas sistem tersebut.

        \vspace{2ex}

  \item \textbf{BAB V Penutup}

        Bab ini berisi kesimpulan dari hasil penelitian serta saran-saran untuk pengembangan lebih lanjut.

\end{enumerate}
