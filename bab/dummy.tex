
\newpage


\subsection{React.js}

React.js merupakan salah satu \emph{library} \emph{JavaScript} yang umum digunakan dalam pengembangan antarmuka pengguna (\emph{user interface}) yang dinamis dan interaktif. React mengadopsi pendekatan berbasis \emph{component}, yaitu unit antarmuka yang terpisah dan modular, sehingga memudahkan proses pengembangan, pemeliharaan, serta pengujian perangkat lunak. Komponen dalam React dapat dibangun melalui dua pendekatan utama, yaitu \emph{class components} dan \emph{functional components}, yang keduanya bertanggung jawab dalam proses \emph{rendering} elemen antarmuka berdasarkan \emph{props} dan \emph{state} \cite{Panjaitan2021}. Efisiensi dalam pembaruan antarmuka didukung oleh penerapan \emph{Virtual DOM}, yang memungkinkan perubahan hanya terjadi pada bagian yang terdampak tanpa memuat ulang seluruh halaman. Pendekatan ini tidak hanya meningkatkan keterbacaan dan perawatan kode, tetapi juga mendukung prinsip \emph{reusability}, yakni penggunaan kembali komponen di berbagai konteks pengembangan. Dengan karakteristik tersebut, React.js menjadi salah satu teknologi dominan dalam pengembangan aplikasi web berbasis \emph{single-page application} (\emph{SPA}) \cite{Panjaitan2021}.

\subsection{Next.js}
Next.js merupakan \emph{framework} berbasis \emph{React} yang dikembangkan untuk mendukung pengembangan aplikasi web \emph{full-stack} secara efisien dan terstruktur. Sebagai ekstensi dari \emph{React}, Next.js menyederhanakan berbagai konfigurasi tingkat rendah seperti \emph{module bundling}, \emph{routing}, dan \emph{code splitting}, yang umumnya memerlukan pengaturan manual dalam proyek \emph{React} murni \cite{Nextjs2024}. Kemampuan ini memungkinkan pengembang untuk fokus pada pengembangan fitur tanpa harus terlibat langsung dalam proses pengelolaan infrastruktur aplikasi. Next.js mengintegrasikan secara native pendekatan \emph{server-side rendering} (SSR) dan \emph{static site generation} (SSG), dua teknik rendering yang krusial dalam optimalisasi performa dan aksesibilitas halaman web. SSR memungkinkan halaman dirender di sisi server sebelum dikirim ke klien, sehingga mengurangi waktu muat awal pada web browswer. Di sisi lain, SSG menghasilkan halaman statis pada waktu kompilasi, yang secara signifikan menurunkan beban server dan mempercepat distribusi konten. Kombinasi kedua pendekatan ini memberikan fleksibilitas tinggi dalam strategi rendering berdasarkan kebutuhan spesifik aplikasi. Dalam konteks pengembangan berbasis komponen, integrasi React dan Next.js mendukung pengembangan antarmuka interaktif dengan performa tinggi. Studi mutakhir menunjukkan bahwa implementasi \emph{framework} seperti Next.js dapat menurunkan latensi interaksi, meningkatkan \emph{user engagement}, serta mendukung skalabilitas dalam arsitektur perangkat lunak modern \cite{Nextjs2024}.

\subsection{Tailwind CSS}

Tailwind CSS merupakan \emph{framework} \emph{utility-first} berbasis CSS yang mengedepankan penggunaan kelas utilitas langsung dalam markup untuk membentuk antarmuka pengguna yang responsif dan terstruktur. Berbeda dengan \emph{framework} tradisional yang menyediakan komponen siap pakai, pendekatan Tailwind memungkinkan fleksibilitas tinggi dan iterasi cepat dalam pengembangan desain kustom \cite{Azhariyah2024}. Integrasinya yang seamless dengan \emph{framework} JavaScript seperti React dan Angular memperkuat efisiensi pengembangan berbasis komponen, sekaligus mengurangi kebutuhan akan file CSS yang besar. Dukungan bawaan terhadap desain responsif menjadikan Tailwind CSS relevan untuk pengembangan aplikasi lintas perangkat dengan konsistensi visual yang tinggi.

\subsection{ShadCN}

ShadCN UI merupakan kumpulan komponen \emph{React} modern yang bersifat \emph{open-source}, namun tidak dikemas sebagai pustaka tradisional yang dapat diinstal melalui \emph{Node Package Manager} (\emph{npm}). Sebaliknya, pengembang menyalin langsung kode sumber komponen ke dalam basis kode proyek, sehingga memberikan fleksibilitas tinggi dalam modifikasi fungsionalitas dan penataan. Komponen-komponen ini menggunakan \emph{Tailwind CSS} untuk styling, dengan desain awal yang konsisten dan minimalis. ShadCN menyediakan beragam komponen antarmuka umum seperti \emph{dialog}, \emph{input}, \emph{checkbox}, dan \emph{table}, serta mendukung integrasi cepat melalui perintah \emph{CLI} untuk menyalin kode ke dalam direktori lokal proyek \cite{Shadcn2024}. Pendekatan ini menghasilkan aplikasi yang lebih ringan dan modular, karena hanya menyertakan komponen yang benar-benar dibutuhkan.


\subsection{WebSocket}
WebSocket adalah protokol komunikasi dua arah yang beroperasi di atas koneksi TCP yang persisten, memungkinkan pertukaran pesan antara klien dan server secara efisien. Protokol ini dirancang untuk mengatasi kekurangan teknologi komunikasi dua arah yang ada sebelumnya, yang sering menggunakan \emph{HTTP} sebagai lapisan transportasi. Dengan memanfaatkan mekanisme \emph{upgrade} dari \emph{HTTP}, WebSocket dapat membuka saluran komunikasi bidirectional yang lebih efisien dibandingkan dengan pendekatan tradisional seperti \emph{polling} atau \emph{long polling} \cite{Fette2011}. Keunggulan utama dari WebSocket terletak pada kemampuannya untuk mengurangi overhead komunikasi, yang sangat penting dalam aplikasi yang memerlukan pembaruan real-time, seperti aplikasi \emph{chat}, permainan daring, dan sistem \emph{IoT}. Penelitian menunjukkan bahwa WebSocket tidak hanya meningkatkan efisiensi komunikasi, tetapi juga memungkinkan pengembangan aplikasi kolaboratif yang lebih responsif dan interaktif \cite{Milsap2019}. Dengan demikian, WebSocket menjadi solusi yang ideal untuk aplikasi yang membutuhkan latensi rendah dan kecepatan tinggi dalam pertukaran data.


