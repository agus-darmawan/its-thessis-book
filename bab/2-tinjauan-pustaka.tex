\chapter{TINJAUAN PUSTAKA}
\label{chap:tinjauanpustaka}



\section{Hasil penelitian/perancangan terdahulu}
Dalam penelitian ini, penulis merujuk pada beberapa studi sebelumnya yang relevan. Penelitian-penelitian tersebut memiliki hubungan dengan topik yang sedang diteliti, sehingga dapat digunakan sebagai dasar untuk penelitian ini.

\subsection{\emph{Autonomous Thermal Vision Robotic System for Victims Recognition}}

Penelitian oleh Cruz Ulloa mengembangkan robot berkaki empat (\emph{quadruped-legged}) menggunakan \emph{Unitree A1} yang dilengkapi dengan kamera termal \emph{Optris Pi640}. Robot ini dirancang untuk mendeteksi korban dalam misi pencarian dan penyelamatan di lingkungan pasca-bencana, dengan memanfaatkan citra termal yang dianalisis menggunakan algoritma \emph{Convolutional Neural Network (CNN)}. Hasil penelitian menunjukkan bahwa sistem ini mampu mencapai akurasi deteksi lebih dari 90\% dalam kondisi ekstrem, seperti lingkungan minim cahaya atau area yang dipenuhi puing bangunan. Robot ini beroperasi secara \emph{teleoperated} dan mampu menjelajah medan yang tidak rata, sehingga sangat membantu tim pencarian dalam menemukan korban yang terperangkap di lokasi bencana \cite{Cruz2021}. Sistem robot dibangun dengan menggunakan \emph{Robot Operating System (ROS) 1 Melodic}, seperti pada gambar \ref{fig:Quadruped dengan kamera termal untuk deteksi korban}

\begin{figure} [H] \centering
  \includegraphics[scale=0.059]{gambar/bab2/unitreea1.png}
  \caption{Diagram sistem \emph{Autonomous Thermal Vision System for Victims Recognition} \cite{Cruz2021}}
  \label{fig:Quadruped  dengan kamera termal untuk deteksi korban}
\end{figure}

Penelitian ini memiliki kesamaan dengan topik yang sedang dikembangkan, yaitu penggunaan robot \emph{quadruped} dan kamera termal untuk memberikan estimasi posisi objek dalam peta, menjadikannya sangat relevan dengan pendekatan yang sedang dirancang.



\subsection{\emph{Image Processing Technique Applied to Electrical
Substations Based on Drones With Thermal Vision
for Predictive Maintenance}}
Penelitian ini mengusulkan penggunaan \emph{VANT} \emph{(Vehículo Aéreo No Tripulado)} atau drone, yang dilengkapi dengan dua jenis kamera: \emph{kamera tradisional} untuk menangkap gambar visual dan \emph{kamera termografik} untuk memperoleh gambar inframerah yang dapat menunjukkan suhu komponen di gardu induk. Drone ini dilengkapi dengan sistem navigasi dan pengendalian yang memungkinkan operasi otonom di sekitar gardu induk. Data gambar yang diambil oleh drone diproses menggunakan teknik \emph{image processing} untuk mengidentifikasi \emph{hot spots} atau titik panas pada komponen gardu induk.

\begin{figure} [H] \centering
  \includegraphics[scale=0.64]{gambar/bab2/drone.png}
  \caption{\emph{GUI} Drone dengan kamera termal untuk pemantauan gardu induk \cite{Prieto2022}}
  \label{fig:Drone dengan kamera termal untuk pemantauan gardu induk}
\end{figure}

Hasil analisis ini dapat digunakan untuk memprediksi potensi kerusakan pada komponen dan mengambil tindakan pencegahan yang diperlukan. Penelitian ini menunjukkan bahwa penggunaan drone dengan kamera termal dapat meningkatkan efisiensi dan akurasi dalam pemantauan gardu induk, serta meminimalkan risiko keselamatan bagi petugas yang harus melakukan inspeksi langsung di lokasi gardu \cite{Prieto2022}. Penelitian ini memiliki kesamaan dengan topik kami yang juga menggunakan kamera termal untuk pemantauan komponen gardu listrik.

\section{Teori/Konsep Dasar}
Subbab ini membahas berbagai teori dan konsep dasar yang menjadi landasan dalam penyusunan tugas akhir. Penjelasan dalam bab ini mencakup berbagai teori yang relevan dan digunakan dalam pelaksanaan penelitian.

\subsection{Gardu Listrik}
\sloppy
Gardu listrik merupakan salah satu komponen vital dalam sistem kelistrikan yang berfungsi sebagai titik penghubung antara pembangkit listrik dan jaringan distribusi. Melalui gardu listrik, aliran energi listrik dapat diatur dan dialirkan secara efisien dari sumber pembangkit menuju konsumen akhir \cite{stevenson1994power}. Proses transmisi dan distribusi energi listrik tidak dapat dilakukan secara aman dan terkontrol tanpa adanya gardu listrik. Secara umum, terdapat beberapa jenis gardu listrik, antara lain gardu induk dan gardu pembangkit. Gardu induk memiliki fungsi utama untuk menurunkan tegangan listrik dari tingkat tinggi, yang digunakan dalam proses transmisi, menjadi tegangan menengah atau rendah agar aman untuk digunakan oleh konsumen Gardu ini umumnya terletak di antara pembangkit dan kawasan konsumen, baik permukiman maupun kawasan industri.

Sementara itu, gardu pembangkit merupakan fasilitas yang berada di dekat sumber energi primer, seperti pembangkit listrik tenaga air (PLTA), pembangkit listrik tenaga uap (PLTU), atau jenis pembangkit lainnya. Gardu ini berfungsi untuk menyalurkan energi listrik yang dihasilkan dari proses konversi energi primer ke jaringan transmisi. Untuk mendukung fungsinya, gardu listrik dilengkapi dengan berbagai peralatan kelistrikan, seperti transformator, pemutus sirkuit, dan isolator untuk menjamin keandalan sistem. Keandalan gardu listrik sangat berpengaruh terhadap kualitas dan kontinuitas pasokan energi listrik \cite{gonen2016electric}.

\subsubsection{2.2.1.1 Gardu Induk}
Gardu induk merupakan salah satu komponen vital dalam sistem tenaga listrik yang berfungsi untuk mengubah, mengatur, dan menyalurkan energi listrik dari pembangkit ke konsumen melalui jaringan transmisi dan distribusi. Fungsinya mencakup menaikkan tegangan listrik dari pembangkit untuk keperluan transmisi jarak jauh, serta menurunkan tegangan sebelum didistribusikan ke konsumen akhir. Tegangan tinggi diperlukan dalam proses transmisi untuk meminimalkan rugi-rugi daya (\emph{losses}) yang terjadi akibat resistansi saluran, sementara penurunan tegangan dibutuhkan agar sesuai dengan batasan keamanan dan kebutuhan perangkat di sisi konsumen. Berdasarkan tingkat tegangan yang diolah, gardu induk diklasifikasikan menjadi beberapa jenis utama, yaitu gardu induk tegangan ekstra tinggi (\emph{GITET}), gardu induk tegangan tinggi (\emph{GIT}), gardu induk tegangan menengah (\emph{GIM}), dan gardu induk tegangan rendah (\emph{GIR}). \emph{GITET} beroperasi pada tegangan di atas 500 kilovolt (kV) dan digunakan untuk transmisi daya dalam skala besar antarpulau atau antardaerah beban yang berjauhan, karena efisiensinya dalam mengurangi kehilangan daya selama proses transmisi. \emph{GIT} menangani tegangan pada rentang 150--500 kV, berfungsi sebagai titik interkoneksi antar sistem transmisi serta sebagai tempat pengaturan pembebanan antara pembangkit dan jaringan utama. Selanjutnya, \emph{GIM} beroperasi pada tegangan menengah antara 20--150 kV, dan biasanya digunakan untuk menghubungkan sistem transmisi ke jaringan distribusi, baik di wilayah industri maupun kawasan pemukiman padat. Terakhir, \emph{GIR} mengelola tegangan di bawah 20 kV, dan bertugas menyuplai energi listrik langsung ke konsumen akhir seperti rumah tangga, pertokoan, dan industri kecil.

Secara umum, setiap gardu induk dilengkapi dengan berbagai peralatan kelistrikan utama, seperti transformator daya untuk konversi tegangan, pemutus sirkuit (\emph{circuit breaker}) untuk proteksi dan pemisahan rangkaian, isolator untuk pemisahan fisik komponen bertegangan, serta perangkat pengukuran dan kontrol lainnya. Selain itu, gardu induk juga dilengkapi dengan sistem proteksi otomatis untuk mendeteksi gangguan dan mencegah kerusakan pada peralatan. Keandalan dan efisiensi operasional gardu induk sangat menentukan kestabilan sistem tenaga listrik secara keseluruhan, terutama dalam menghadapi fluktuasi beban, kondisi darurat, maupun pemeliharaan jaringan\cite{gonen2016electric}.



\subsubsection{2.2.1.2 Transformator}

Transformator merupakan salah satu komponen utama dalam gardu listrik yang berfungsi untuk mengubah tingkat tegangan sesuai kebutuhan dalam proses transmisi maupun distribusi energi listrik. Transformator bekerja berdasarkan prinsip induksi elektromagnetik, di mana energi listrik dialirkan melalui dua kumparan yang terpisah secara galvanis namun terhubung melalui inti magnetik. Dalam sistem tenaga listrik, transformator tidak hanya digunakan untuk mengatur level tegangan, tetapi juga untuk meningkatkan efisiensi dan keamanan sistem. Salah satu jenis transformator yang umum digunakan pada gardu listrik adalah transformator daya. Transformator ini berfungsi untuk menyesuaikan tegangan agar sesuai dengan kebutuhan sistem, serta meminimalkan rugi-rugi daya selama proses transmisi jarak jauh. Berdasarkan fungsinya, transformator daya dibedakan menjadi dua jenis utama, yaitu transformator \emph{step-up} dan transformator \emph{step-down}. Transformator \emph{step-up} digunakan di sisi pembangkit untuk menaikkan tegangan listrik ke tingkat yang lebih tinggi agar proses transmisi menjadi lebih efisien dan rugi daya dapat dikurangi. Sebaliknya, transformator \emph{step-down} digunakan di sisi distribusi untuk menurunkan tegangan ke tingkat yang lebih rendah agar aman digunakan oleh konsumen akhir, seperti rumah tangga dan industri.

Selain transformator daya, gardu listrik juga dilengkapi dengan transformator arus atau \emph{current transformer} (CT). CT berfungsi untuk mengubah arus listrik yang besar pada sistem tenaga menjadi arus yang lebih kecil dan aman untuk keperluan pengukuran, pengendalian, dan proteksi. Dengan menggunakan CT, perangkat seperti relai proteksi, sistem pemantauan, dan peralatan metering dapat mengamati kondisi arus tanpa harus langsung berhubungan dengan arus utama yang tinggi dan berbahaya. Transformator arus dirancang dengan akurasi dan rasio transformasi tertentu untuk memastikan bahwa sinyal arus yang dikirimkan ke peralatan sekunder tetap mencerminkan kondisi nyata sistem. Dari sisi termal, transformator daya umumnya dirancang untuk beroperasi pada suhu maksimum antara 105\textdegree{}C hingga 110\textdegree{}C, tergantung pada kelas isolasi termalnya, seperti kelas A (105\textdegree{}C), B (130\textdegree{}C), atau F (155\textdegree{}C). Sementara itu, transformator arus (CT) biasanya dirancang untuk beroperasi pada rentang suhu antara 70\textdegree{}C hingga 90\textdegree{}C, tergantung pada standar pabrikan dan klasifikasi aplikasinya. Meskipun arus yang masuk ke CT merupakan arus tinggi, arus keluarannya kecil sehingga beban termal yang diterima relatif rendah. Namun demikian, CT tetap memerlukan sistem ventilasi atau pendinginan yang memadai agar kinerja pengukuran tetap stabil dan umur operasional peralatan dapat terjaga. Pengelolaan suhu operasional yang baik pada kedua jenis transformator ini menjadi faktor krusial untuk menjaga keandalan dan efisiensi sistem tenaga secara keseluruhan \cite{stevenson1994power}.


\subsubsection{2.2.1.3 Pemutus Sirkuit (\emph{Circuit Breaker})}
\emph{Pemutus sirkuit} adalah perangkat yang berfungsi untuk melindungi sistem kelistrikan dari arus lebih (\emph{overcurrent}) dan hubung singkat (\emph{short circuit}). \emph{Pemutus sirkuit} dapat secara otomatis memutuskan aliran listrik ketika terdeteksi adanya gangguan, sehingga mencegah kerusakan pada peralatan dan menjaga keselamatan sistem. Terdapat berbagai jenis \emph{pemutus sirkuit}, termasuk \emph{pemutus sirkuit otomatis} (\emph{automatic circuit breaker}) dan \emph{pemutus sirkuit manual} (\emph{manual circuit breaker}), yang masing-masing memiliki aplikasi dan karakteristik yang berbeda Pemeliharaan dan pengujian berkala terhadap \emph{pemutus sirkuit} sangat penting untuk memastikan bahwa perangkat ini berfungsi dengan baik saat dibutuhkan. Suhu operasi \emph{pemutus sirkuit} biasanya berkisar antara -25°C hingga 55°C, dan overheating dapat terjadi jika suhu melebihi 85°C, yang dapat menyebabkan kerusakan pada mekanisme pemutus \cite{Ilomets2020}.


\subsubsection{2.2.1.4 \emph{Arrester}}
\emph{Arrester}, atau \emph{lightning arrester}, merupakan perangkat pelindung dalam sistem kelistrikan yang berfungsi untuk mengamankan peralatan dari lonjakan tegangan akibat sambaran petir atau gangguan tegangan lebih sesaat. \emph{Arrester} bekerja dengan cara mengalihkan arus lebih tersebut langsung ke \emph{ground}, sehingga mencegah terjadinya kerusakan pada peralatan penting seperti transformator, pemutus sirkuit, dan komponen lainnya di gardu listrik. Pemeliharaan rutin terhadap \emph{arrester} sangat penting dilakukan untuk memastikan performanya tetap optimal. Salah satu parameter penting yang harus dipantau adalah suhu operasi perangkat. Umumnya, \emph{arrester} dirancang untuk beroperasi pada suhu lingkungan antara -40\textdegree{}C hingga 60\textdegree{}C. Jika suhu operasional melebihi batas maksimum tersebut, maka dapat terjadi kerusakan permanen pada komponen internal arrester yang berakibat pada menurunnya efektivitas perlindungan sistem secara keseluruhan \cite{Kartika2022}.

\subsubsection{2.2.1.5 Disconnector}
\emph{Disconnector} merupakan perangkat yang berfungsi untuk memisahkan suatu bagian dari sistem kelistrikan guna keperluan pemeliharaan atau perbaikan, tanpa memengaruhi bagian lain dari jaringan. Pemeliharaan serta pengujian secara berkala terhadap \emph{disconnector} sangat krusial untuk memastikan bahwa perangkat ini dapat berfungsi secara optimal saat dibutuhkan. Suhu operasi \emph{disconnector} umumnya berada dalam kisaran 20°C hingga 40°C. Namun, apabila suhu meningkat melebihi 85°C, dapat terjadi overheating yang berpotensi menyebabkan kegagalan fungsi \cite{Henriana2022}.


\subsubsection{2.2.1.6 Busbar}
\emph{Busbar} adalah komponen dalam gardu listrik yang berfungsi sebagai penghubung antara berbagai peralatan listrik. \emph{Busbar} memungkinkan distribusi arus listrik yang efisien dan aman di dalam gardu. \emph{Busbar} biasanya terbuat dari bahan konduktif yang baik, seperti tembaga atau aluminium, dan dirancang untuk menampung arus listrik dalam jumlah besar. Pemeliharaan dan pengujian \emph{busbar} secara rutin diperlukan untuk mencegah kerusakan dan memastikan keandalan sistem distribusi listrik. Suhu operasi \emph{busbar} dapat bervariasi, tetapi umumnya tidak boleh melebihi 90°C untuk mencegah overheating yang dapat merusak isolasi dan struktur busbar itu sendiri \cite{Telaumbanua2024}.

\subsubsection{2.2.1.7 Isolator}
\emph{Isolator} adalah perangkat yang berfungsi untuk memisahkan bagian dari sistem kelistrikan, sehingga mencegah arus listrik mengalir ke bagian yang tidak diinginkan. \emph{Isolator} digunakan untuk menjaga keamanan dan keandalan sistem kelistrikan, terutama saat pemeliharaan dilakukan. \emph{Isolator} dirancang untuk menahan tegangan tinggi dan memiliki karakteristik dielektrik yang baik. Pemeliharaan \emph{isolator} juga penting untuk memastikan bahwa tidak ada kebocoran arus yang dapat menyebabkan kerusakan pada peralatan lainnya. Suhu operasi \emph{isolator} biasanya berkisar antara -30°C hingga 50°C, dan overheating dapat terjadi jika suhu melebihi 70°C, yang dapat mengakibatkan kerusakan pada material isolasi \cite{Moreno2017}.

\subsection{DeepRobotics Jueying X30 Pro}
DeepRobotics Jueying X30 Pro adalah robot berkaki empat (\textit{quadruped}) generasi terbaru yang dikembangkan oleh perusahaan teknologi robotik DeepRobotics di Tiongkok. Dirancang untuk berbagai aplikasi industri seperti inspeksi fasilitas, penyelamatan darurat, eksplorasi lingkungan ekstrem, serta pemantauan di berbagai medan, robot ini mengintegrasikan teknologi berbagai sensor untuk navigasi dan persepsi lingkungan. Dalam mendukung navigasi dan pemetaan lingkungan, X30 Pro dilengkapi dengan empat sensor LiDAR 360° Livox, dua di bagian depan dan dua di bagian belakang. Selain itu, X30 Pro dilengkapi dengan IMU (\textit{Inertial Measurement Unit}) Yesense YIS100-V, yang menggabungkan akselerometer dan giroskop untuk mengukur percepatan linear dan kecepatan sudut. Dengan desain industrial Jueying X30 Pro mampu bergerak secara cepat dan stabil di berbagai jenis permukaan. Kemampuan tersebut didukung oleh integrasi teknologi persepsi \emph{multimodal} dan sistem kendali gerak adaptif, sehingga memungkinkan navigasi otonom dalam kondisi lingkungan yang kompleks. Spesifikasi teknis utama dari Jueying X30 Pro ditampilkan pada Tabel~\ref{tab:spesifikasiX30}.


\begin{table}[H]
	\centering
	\caption{Spesifikasi Teknis Deep Robotics X30} \
	\label{tab:spesifikasiX30}
	\renewcommand{\arraystretch}{1.2}
	\begin{tabular}{|p{5.5cm}|p{7cm}|}
		\hline
		\textbf{Parameter}            & \textbf{Spesifikasi}                                              \\
		\hline
		Dimensi (P × L × T)           & 1000 mm × 585 mm × 470 mm                                         \\
		\hline
		Bobot (berat bersih)          & 56--59 kg                                                         \\
		\hline
		Kecepatan Maksimal            & 4 m/s                                                             \\
		\hline
		Kemampuan Menanjak            & Hingga 45\textdegree{} (termasuk tangga terbuka)                  \\
		\hline
		Ketinggian Rintangan Maksimal & $\geq$ 20 cm                                                      \\
		\hline
		Rentang Suhu Operasional      & --20\textdegree{}C hingga +55\textdegree{}C                       \\
		\hline
		Perlindungan Lingkungan       & IP67 (tahan air dan debu)                                         \\
		\hline
		Daya Tahan Baterai            & 2,5--4 jam (dengan beban penuh)                                   \\
		\hline
		Sistem Pengisian Daya         &  (\emph{Auto-charging}), \emph{manual charging} \\
		\hline
		Port Daya Eksternal           & 24V hingga 240 W, 12V hingga 36 W, daya total maksimal 350 W      \\
		\hline
		Antarmuka Komunikasi          & Ethernet, USB 2.0/3.0, dan Wi-Fi                                  \\
		\hline
	\end{tabular}
\end{table}


Untuk mendukung operasi otonom dan efisien, \emph{DeepRobotics Jueying X30 Pro} dilengkapi dengan tiga unit komputer onboard yang masing-masing bertanggung jawab atas fungsi gerak (\emph{motion}), persepsi (\emph{perception}), dan navigasi (\emph{navigation}). Arsitektur ini memungkinkan pemrosesan paralel dan spesialisasi tugas, meningkatkan kinerja robot dalam lingkungan yang kompleks dan dinamis. Ketiga komputer ini terhubung melalui Ethernet seperti pada Gambar \ref{fig:network_architecture_x30pro}.


\begin{figure}[H]
  \centering
  \includegraphics[width=0.65\textwidth]{gambar/bab2/network-x30.png}
  \caption{Arsitektur Jaringan Internal \emph{DeepRobotics Jueying X30 Pro} \cite{deeprobotics_jueyingx30}}
  \label{fig:network_architecture_x30pro}
\end{figure}


Arsitektur modular ini tidak hanya meningkatkan efisiensi operasional tetapi juga memudahkan integrasi dengan sistem eksternal dan pembaruan perangkat lunak. Robot ini menggunakan \emph{Robot Operating System (ROS) Noetic Ninjameys} dan dijalankan menggunakan sistem operasi Ubuntu 20.04, yang semakin memperluas potensi integrasi dan pengembangan aplikasi berbasis \emph{open-source}.


\subsection{Hikvision HM-TD5528T-15/W}
Hikvision HM-TD5528T-15/W merupakan kamera pengawas tipe \emph{Thermographic Thermal and Optical Bi-spectrum Network Mini Positioning System} yang dirancang untuk pemantauan area kritis seperti gardu distribusi, fasilitas industri, dan zona rawan kebakaran. Kamera ini menggabungkan sensor termal dan sensor optik beresolusi tinggi dalam satu unit, serta didukung unit pemrosesan berbasis GPU yang memungkinkan pemantauan suhu dan visual secara \emph{real-time}. Kamera ini menggunaka modul termal sensor \emph{Vanadium Oxide Uncooled Focal Plane Arrays} dengan resolusi \(256 \times 192\) piksel dan \emph{pixel pitch} \(12~\mu\text{m}\). Kamera ini memiliki sensitivitas termal tinggi dengan \emph{NETD} \(\leq 35~\text{mK}\) pada suhu \(25^\circ\text{C}\), dan dilengkapi lensa dengan panjang fokus 15~mm serta aperture F1.0. Sudut pandang termal sebesar \(11.69^\circ \times 8.78^\circ\) memberikan cakupan area yang cukup sempit namun presisi, dengan IFOV \(0.8~\text{mrad}\) dan jarak fokus minimum 2{,}5~m. Zoom digital tersedia hingga \(8 \times\).

Modul optik kamera ini menggunakan sensor CMOS \(1/2.8^{\prime\prime}\) progresif dengan resolusi hingga 4~MP (\(2688 \times 1520\) piksel). Rentang panjang fokus optik adalah 4{,}8--153~mm, dengan \emph{optical zoom} hingga \(32 \times\) dan \emph{digital zoom} hingga \(16 \times\). Kamera ini juga mendukung fitur \emph{WDR} 120~dB, \emph{optical defog}, serta \emph{picture-in-picture} yang memungkinkan tampilan kanal termal ditampilkan bersamaan di kanal optik. Dalam hal pengukuran suhu, kamera ini menyediakan hingga 273 skenario yang masing-masing dapat memiliki hingga 21 aturan (10 titik, 10 area, dan 1 garis). Rentang suhu yang dapat diukur adalah dari \(-20^\circ\text{C}\) hingga \(550^\circ\text{C}\), dengan akurasi \(\pm 2^\circ\text{C}\) atau \(\pm 2\%\), tergantung nilai yang lebih besar. Kamera ini merupakan jenis kamera thermal CCTV yang tidak memiliki konfigurasi nilai emisivitas secara langsung. Oleh karena itu, koreksi emisivitas perlu dilakukan secara eksternal, tergantung jenis permukaan target (contoh: logam teroksidasi sekitar 0{,}80, kulit manusia 0{,}95, keramik 0{,}90). Dari sisi integrasi sistem, perangkat ini mendukung berbagai protokol jaringan seperti IPv4/IPv6, RTSP, HTTP/HTTPS, ISAPI, dan ONVIF (Profile S, G, T). Penyimpanan data dapat dilakukan melalui kartu microSD hingga 128~GB dan NAS, dengan dukungan fitur \emph{Automatic Network Replenishment} (ANR). Kamera juga menyediakan antarmuka komunikasi seperti Ethernet RJ45 dan RS-485, serta kompatibel dengan perangkat lunak pemantauan seperti iVMS-4200 dan Hik-Connect \cite{hikmicro_hmptz}.


\subsection{Doublecom DB6000FR-ANS}

Doublecom DB6000FR-ANS merupakan terminal nirkabel industri kelas luar ruang yang dirancang untuk lingkungan dengan mobilitas tinggi seperti kendaraan AGV, robot pintar, drone, dan sistem logistik otomatis. Perangkat ini mendukung standar komunikasi nirkabel IEEE 802.11a/n dan mengimplementasikan teknologi MIMO 2T2R dengan arsitektur dua antena pengirim dan penerima. DB6000FR-ANS memiliki bandwidth maksimum sebesar 300~Mbps dan daya pancar maksimum hingga 1000~mW. Perangkat ini juga dilengkapi dengan antarmuka RF eksternal (2×SMA) yang memungkinkan penggunaan antena eksternal dengan berbagai karakteristik gain untuk meningkatkan jangkauan hingga lebih dari 1~km dalam kondisi tertentu. Sensitivitas penerimaannya mencapai \(-96~\text{dBm}\), yang menjadikannya cocok untuk komunikasi jarak jauh dan lingkungan padat interferensi.

Fitur utama lainnya mencakup dukungan \emph{Fast Roaming} (perpindahan AP dengan jeda kurang dari 50~ms), teknologi TDMA untuk efisiensi dalam komunikasi multipoint, dan berbagai fungsi jaringan seperti VLAN, NAT, QoS, VPN, serta firewall. Selain itu, perangkat ini mendukung bandwidth fleksibel (5~MHz, 10~MHz, 20~MHz, dan 40~MHz), serta pengaturan spektrum dinamis melalui DFS. Secara perangkat keras, DB6000FR-ANS dilengkapi prosesor 600~MHz, RAM 64~MB, dan antarmuka jaringan 10/100/1000BASE-T (RJ45). Desainnya tahan terhadap guncangan, antistatik, serta mampu bekerja dalam rentang suhu ekstrem antara \(-40^\circ\text{C}\) hingga \(75^\circ\text{C}\). Perangkat ini juga mendukung manajemen lokal maupun cloud melalui platform aplikasi dan menyediakan indikator sinyal LED, port POE 12--24~VDC, serta konsumsi daya maksimum hanya 9~W \cite{doublecom_db6000frans}.

\subsection{Doublecom DB6000ANLT90-FR}
Doublecom DB6000ANLT90-FR merupakan perangkat nirkabel luar ruang kelas operator (\emph{carrier-grade}) yang dirancang untuk menjadi titik pusat dalam komunikasi nirkabel berbasis sektor. Perangkat ini menggabungkan protokol IEEE 802.11a/n dengan antena dual-pol 90° terintegrasi, bekerja dalam pita frekuensi 5.8~GHz bebas lisensi, mencakup rentang 5150--5850~MHz, dan memiliki gain antena sebesar 18~dBi. DB6000ANLT90-FR menggunakan arsitektur MIMO 2T2R dengan kecepatan transmisi maksimum hingga 300~Mbps. Perangkat ini dirancang untuk komunikasi \emph{point-to-multipoint} dalam radius hingga 5~km, dan dapat mentransmisikan lebih dari 10 kanal video HD dengan bandwidth total lebih dari 100~Mbps. Ini dimungkinkan berkat penerapan teknologi TDMA yang mengatur alokasi waktu transmisi antar titik secara efisien.

Dari sisi perangkat keras, perangkat ini dilengkapi CPU 600~MHz, RAM 64~MB, serta satu port jaringan RJ-45 berkecepatan 10/100/1000BASE-T. Power supply dapat disuplai melalui adaptor DC 12--24~V atau AC 200--300~V, serta mendukung PoE. Fitur manajemen jaringan meliputi DHCP, VLAN, NAT, QoS, firewall, packet sniffing, dan analisis lalu lintas secara langsung. Dari sisi ketahanan lingkungan, perangkat ini dilengkapi proteksi IP67 yang tahan air, angin, debu, dan paparan sinar matahari langsung. Perangkat dapat beroperasi dalam suhu \(-40^\circ\text{C}\) hingga \(75^\circ\text{C}\), kelembapan hingga 95\% (non-kondensasi), serta memiliki desain tahan guncangan dan disipasi panas. Konsumsi daya maksimum perangkat ini sekitar 14~W, dan dapat dipasang menggunakan bracket tipe-L atau sistem penjepit pipa berdiameter \(\phi 30{-}55~\text{mm}\) \cite{doublecom_db6000anlt90}.

\subsection{Moxa EDS-205}

Moxa EDS-205 adalah perangkat \emph{unmanaged Ethernet switch} industri dengan lima port RJ45 10/100BASE-T(X) yang mendukung mode \emph{auto negotiation} dan \emph{auto MDI/MDI-X}. Perangkat ini cocok untuk pengaplikasian di lingkungan industri karena mendukung standar IEEE 802.3, 802.3u, dan 802.3x, serta memiliki kemampuan proteksi terhadap \emph{broadcast storm} EDS-205 menggunakan metode pemrosesan \emph{store and forward}, memiliki ukuran tabel MAC sebesar 2K, dan buffer paket sebesar 512~kbit. Perangkat ini beroperasi pada tegangan masukan 12--48~VDC, dengan konsumsi arus sekitar 0.12~A pada 24~VDC dan proteksi terhadap polaritas terbalik serta arus lebih (\(1.1~\text{A}\)). Dari sisi fisik, perangkat ini memiliki dimensi \(24.9 \times 100 \times 86.5~\text{mm}\), berat 135~g, dan dipasang melalui rel DIN. Perangkat ini dirancang dengan peringkat IP30 dan dapat beroperasi dalam suhu \(-10^\circ\text{C}\) hingga \(60^\circ\text{C}\), serta disertifikasi dengan berbagai standar keselamatan dan kompatibilitas elektromagnetik (seperti UL 508, EN 62368-1, EN 55032/35, IEC 61000-4-x). Perangkat ini sangat andal untuk kebutuhan komunikasi data industri yang tidak memerlukan konfigurasi kompleks \cite{moxa_eds205}.

\subsection{\emph{Robot Operating System (ROS)}}

\emph{Robot Operating System (ROS)} merupakan sebuah \emph{framework} perangkat lunak \emph{open-source} yang dikembangkan untuk memfasilitasi pengembangan aplikasi robotik yang modular dan skalabel. ROS menyediakan seperangkat \emph{tools} dan \emph{libraries} yang dirancang untuk mengelola berbagai fungsi utama dalam sistem robot, seperti kendali aktuator, pemrosesan data sensor, navigasi, dan perencanaan lintasan. Salah satu keunggulan utama ROS adalah arsitekturnya yang terdistribusi, memungkinkan pengembang untuk membagi sistem kompleks ke dalam unit-unit proses yang saling terhubung, dikenal sebagai \emph{node}. ROS mendukung beragam aplikasi, mulai dari robot industri hingga sistem otonom.Berikut merupakan komponen dan konsep fundamental yang membentuk arsitektur dan ekosistem ROS:

\subsubsection{\emph{2.2.7.1 Node}}
\emph{Node} adalah unit eksekusi dasar dalam ROS yang merepresentasikan sebuah proses independen untuk menjalankan tugas tertentu dalam sistem robot. Setiap \emph{node} memiliki tanggung jawab spesifik, seperti pembacaan data sensor atau pengendalian aktuator, dan dapat berkomunikasi dengan \emph{node} lainnya melalui mekanisme seperti \emph{topic}, \emph{service}, atau \emph{action}. Pendekatan ini memungkinkan sistem bekerja secara paralel dan modular.

\subsubsection{\emph{2.2.7.2 Topic}}
\emph{Topic} merupakan mekanisme komunikasi berbasis model \emph{publish/subscribe} yang memungkinkan pertukaran data antar \emph{node} secara asinkron. \emph{Node} yang berperan sebagai \emph{publisher} akan mengirimkan data ke suatu \emph{topic}, sedangkan \emph{subscriber} akan menerima data dari \emph{topic} tersebut. Data yang ditransmisikan berbentuk \emph{message} yang memiliki struktur dan tipe tertentu yang telah ditentukan.

\subsubsection{\emph{2.2.7.3 Service}}
\emph{Service} merupakan bentuk komunikasi sinkron antara dua \emph{node}, di mana satu \emph{node} mengirimkan permintaan (\emph{request}) dan \emph{node} lain memberikan tanggapan (\emph{response}) secara langsung. Mekanisme ini cocok untuk interaksi yang bersifat langsung dan satu kali, seperti pembacaan status atau perubahan konfigurasi.

\subsubsection{\emph{2.2.7.4 Action}}
\emph{Action} digunakan untuk menangani proses yang memerlukan waktu eksekusi yang relatif lama dan memungkinkan pemantauan progres secara kontinu. Berbeda dengan \emph{service} yang bersifat sinkron, \emph{action} menyediakan kemampuan komunikasi dua arah secara asinkron dengan dukungan umpan balik berkala (\emph{feedback}) serta notifikasi penyelesaian (\emph{result}).

\subsubsection{\emph{2.2.7.5 Launch File}}
\emph{Launch file} adalah berkas berformat XML yang digunakan untuk meluncurkan beberapa \emph{node} dan parameter konfigurasi secara bersamaan. File ini memungkinkan pengelolaan sistem ROS secara lebih efisien dan konsisten, terutama dalam pengujian dan pengoperasian sistem yang melibatkan banyak komponen secara paralel.

\subsubsection{\emph{2.2.7.6 Bag File}}
\emph{Bag file} merupakan format file dalam ROS yang digunakan untuk merekam dan memutar ulang data dari \emph{topic}. Fasilitas ini sangat berguna dalam pengujian, analisis, dan debugging karena memungkinkan simulasi ulang dari data historis tanpa perlu menjalankan robot secara langsung.

\subsubsection{\emph{2.2.7.7 Package}}
\emph{Package} adalah unit organisasi utama dalam ROS yang mengelompokkan \emph{node}, pustaka, dan file konfigurasi ke dalam satu kesatuan logis. Paket memfasilitasi pengembangan perangkat lunak yang terstruktur, modular, serta memudahkan distribusi dan pemeliharaan proyek dalam skala besar.

\subsubsection{\emph{2.2.7.8 Parameter}}
\emph{Parameter} adalah variabel konfigurasi yang disimpan dalam \emph{parameter server} ROS. Parameter ini memungkinkan \emph{node} untuk berbagi nilai konfigurasi secara global dalam sistem. Dengan memanfaatkan parameter, perubahan konfigurasi dapat dilakukan secara dinamis tanpa perlu memodifikasi kode sumber.

\subsubsection{\emph{2.2.7.9 Workspace}}
\emph{Workspace} adalah direktori kerja yang digunakan untuk mengelola pengembangan proyek ROS, termasuk penyimpanan kode sumber, hasil kompilasi, dan instalasi. Struktur \emph{workspace} umumnya mencakup folder \texttt{src}, \texttt{devel}, dan \texttt{install}, serta mendukung penggunaan \emph{catkin} sebagai sistem \emph{build} utama.

\subsubsection{\emph{2.2.7.10 Transformation (tf)}}
\emph{Transformation} adalah pustaka dalam ROS yang digunakan untuk melacak dan mengelola transformasi spasial antar kerangka koordinat (\emph{frame}) dalam sistem robot. Dengan \emph{tf}, pengembang dapat melakukan transformasi posisi dan orientasi antar objek atau sensor berdasarkan referensi waktu nyata. Pustaka ini menyediakan antarmuka seperti \texttt{tf::TransformListener} untuk mendengarkan transformasi dan \texttt{tf::TransformBroadcaster} untuk mempublikasikan transformasi antar \emph{frame}. Fitur ini sangat krusial dalam aplikasi navigasi, pemetaan, dan manipulasi objek, di mana presisi spasial menjadi aspek fundamental \cite{ros_noetic}.

\subsection{\emph{Localization dan Mapping}}

 \textit{ Localization and Mapping} berperan penting dalam sistem robotik dan kendaraan otonom, terutama dalam membangun peta lingkungan sekaligus menentukan posisi robot secara presisi. Seiring berkembangnya metode pemrosesan data LIDAR dan sensor IMU, berbagai pendekatan telah dikembangkan untuk meningkatkan akurasi, efisiensi komputasi, dan ketahanan terhadap dinamika lingkungan. 

\subsubsection{2.2.8.1 NanoGICP}

NanoGICP adalah algoritma registrasi cloud point ringan yang dimodifikasi dari \emph{Generalized Iterative Closest Point} (GICP). Metode ini menggabungkan pendekatan matriks kovarian dalam GICP dengan struktur data KD-tree yang dioptimalkan untuk pemrosesan cepat. NanoGICP memanfaatkan estimasi kovarian yang teraproksimasi untuk menekan kompleksitas komputasi, namun tetap mempertahankan ketahanan terhadap data nois. Hal ini menjadikannya ideal untuk sistem \emph{LiDAR-based SLAM} pada perangkat dengan keterbatasan komputasi, seperti robot berukuran kecil atau \emph{embedded systems} \cite{koide2021nanogicp}.


\subsubsection{2.2.8.2 Quatro}

\emph{Quatro} merupakan sistem lokalisasi berbasis \emph{LiDAR} yang mengintegrasikan empat strategi utama: \emph{scan-to-scan registration}, \emph{scan-to-map matching}, ekstraksi fitur bidang (\emph{plane-feature}), dan kompensasi deformasi waktu (\emph{time-deformed registration}). Kombinasi ini memungkinkan estimasi pose yang lebih akurat dan stabil dalam lingkungan yang dinamis serta berkontur kompleks. Quatro terbukti memiliki performa unggul pada berbagai benchmark dataset seperti KITTI dan NCLT, serta mampu berjalan \emph{real-time} dengan efisiensi komputasi tinggi \cite{kim2022quatro}.

\subsubsection{2.2.8.3 \emph{Direct LiDAR-Inertial Odometry (DLIO)}}`
Algoritma \emph{Direct LiDAR-Inertia'l Odometry} (DLIO) menawarkan solusi dengan pendekatan \emph{coarse-to-fine} untuk menghasilkan koreksi gerakan waktu kontinu secara lebih akurat. DLIO memanfaatkan persamaan analitik yang dirancang untuk memperbaiki setiap titik data secara paralel dan efisien, sekaligus mengintegrasikan data LiDAR dan \emph{Inertial Measurement Unit} (IMU) secara ketat untuk menghasilkan estimasi keadaan robot yang lebih akurat. Pendekatan ini memungkinkan koreksi distorsi pada \emph{point cloud} secara \emph{real-time}, bahkan dalam kondisi pergerakan dinamis yang kompleks \cite{chen2022dlio}.


\begin{figure} [H] \centering
  \includegraphics[scale=0.6]{gambar/bab2/dlio_Arch.png}
  \caption{Arsitektur \emph{Direct LiDAR-Inertial Odometry (DLIO)} \cite{chen2022dlio}}
  \label{fig:DLIO Architecture}
\end{figure}

Keunggulan DLIO tidak hanya terletak pada akurasi tinggi dalam estimasi posisi, tetapi juga pada efisiensi komputasi yang lebih baik dibandingkan algoritma-algoritma terkini lainnya seperti LIO-SAM dan FAST-LIO2. DLIO menggabungkan koreksi gerakan dan pembuatan \emph{prior} dalam satu langkah, menghilangkan kebutuhan akan \emph{scan-to-scan matching} yang biasanya memakan waktu. Melalui eksperimen pada \emph{dataset} publik dan \emph{dataset} lapangan yang dikumpulkan secara mandiri, DLIO menunjukkan keunggulan dalam menghasilkan peta 3D yang detail dengan kesalahan posisi minimum. Sistem ini menjadi alternatif yang menjanjikan untuk platform robot bergerak, termasuk drone, dalam navigasi dan pemetaan di lingkungan yang tidak terstruktur.



\subsubsection{2.2.8.4 \emph{Fast-LIO2}: \emph{Fast Direct LiDAR-Inertial Odometry}}

\emph{Fast-LIO2} adalah sistem \emph{LiDAR-Inertial Odometry} (\emph{LIO}) yang dirancang untuk efisiensi dan akurasi tinggi. Sistem ini menggunakan \emph{tightly-coupled iterated Kalman filter} dan dua inovasi utama: pertama, registrasi langsung titik-titik \emph{LiDAR} mentah ke peta voxel tanpa ekstraksi fitur eksplisit, sehingga menangkap detail lingkungan secara lebih halus dan fleksibel terhadap berbagai jenis \emph{LiDAR}; kedua, struktur data \emph{incremental k-d tree} (\emph{ikd-Tree}) yang mendukung pembaruan peta secara dinamis dan efisien, serta unggul dibanding \emph{octree}, \emph{R*-tree}, dan \emph{nanoflann}.

\begin{figure}[H]
    \centering
    \includegraphics[width=1\textwidth]{gambar/bab2/fast-lio2.png}
    \caption{Diagram sistem \emph{Fast-LIO2}. \cite{xu2022fastlio}}
    \label{fig:fastlio2}
\end{figure}

\emph{Fast-LIO2} terbukti unggul dalam 19 sekuen dataset publik dan mampu berjalan hingga 100 Hz, bahkan dalam kondisi ekstrem seperti rotasi hingga $1000^\circ/\text{s}$ \cite{xu2022fastlio}.

\subsubsection{2.2.8.5 Fast-LIO-SAM sebagai Generator Peta}

\emph{Fast-LIO-SAM} adalah pengembangan dari \emph{Fast-LIO2} yang menggabungkan estimasi odometri berbasis \emph{LiDAR-Inertial Odometry} (\emph{LIO}) dengan backend optimisasi graf berbasis \emph{Smoothing and Mapping} (\emph{SAM}). Sistem ini memisahkan proses estimasi pose (\emph{frontend}) dan pemetaan serta koreksi \emph{loop closure (backend)}, sehingga mampu menghasilkan peta 3D yang konsisten dan akurat dalam skala besar. Salah satu keunggulan utama dari \emph{Fast-LIO-SAM} adalah kemampuannya untuk membangun dan menyimpan peta lingkungan dalam format data \texttt{.bag} (ROS bag file). Peta ini berisi informasi spasial hasil integrasi data \emph{LiDAR} dan \emph{IMU} yang telah dioptimasi secara global, dan dapat digunakan kembali untuk proses lokalisasi berulang di waktu yang berbeda tanpa perlu membangun ulang peta. Struktur peta voxel yang digunakan dalam \emph{backend SAM} membuat sistem ini efisien dalam konsumsi memori dan mendukung pemrosesan \emph{real-time}. Oleh karena itu, \emph{Fast-LIO-SAM} ideal digunakan sebagai tahap awal dalam pipeline sistem navigasi otonom, terutama ketika diperlukan peta dasar (\emph{prior map}) untuk aplikasi lokalisasi jangka panjang \cite{xu2022fastlio}.

\subsubsection{2.2.8.6 Fast-LIO Localization QN: Integrasi Quatro dan NanoGICP}

\emph{Fast-LIO Localization QN} adalah sistem lokalisasi berbasis peta yang dirancang untuk bekerja dengan peta hasil dari sistem seperti \emph{Fast-LIO-SAM}. Peta yang telah dibangun dan disimpan dalam format \texttt{.bag} melalui proses SLAM sebelumnya akan digunakan sebagai referensi untuk melakukan lokalisasi kembali terhadap data \emph{LiDAR} baru. Pendekatan ini memungkinkan navigasi yang presisi dan efisien tanpa harus melakukan pemetaan ulang secara terus-menerus.Arsitektur QN menggabungkan modul odometri dari \emph{Fast-LIO2} dengan dua komponen utama dalam proses \emph{map matching}, yaitu \emph{Quatro} dan \emph{NanoGICP}. \emph{Quatro} digunakan untuk melakukan registrasi global cepat, yang memberikan perkiraan awal transformasi posisi sensor terhadap peta. Hal ini sangat penting dalam skenario dengan deviasi posisi awal besar atau ketika sistem baru aktif. Setelah diperoleh transformasi awal, \emph{NanoGICP} digunakan untuk proses refinemen lokal. Algoritma ini merupakan kombinasi dari \emph{FastGICP} dan \emph{NanoFLANN}, yang memungkinkan pencocokan lokal secara presisi tinggi dengan beban komputasi rendah. Integrasi ini memungkinkan proses lokalisasi yang cepat, akurat, dan tahan terhadap noise lingkungan. Karena QN tidak melakukan pemetaan ulang, sistem ini sangat cocok untuk digunakan dalam skenario di mana peta telah tersedia, seperti area industri, fasilitas logistik, gudang, dan bangunan besar dengan struktur tetap. Sistem ini juga mempercepat waktu inisialisasi dan mengurangi konsumsi sumber daya komputasi, karena hanya fokus pada lokalisasi berbasis peta referensi yang diberikan\cite{fastlio2023qnloc}.


\subsection{\emph{Navigation}}
\emph{Path navigation} merupakan komponen kunci dalam sistem robotik bergerak otonom. Komponen ini bertugas memastikan bahwa robot dapat mengikuti jalur yang telah direncanakan sekaligus mampu bereaksi terhadap dinamika lingkungan seperti rintangan statis maupun bergerak. Pendekatan navigasi umumnya dibagi menjadi dua: pengendalian jalur (\emph{path following}) dan penghindaran rintangan (\emph{obstacle avoidance}). Dalam subbagian ini dibahas tiga pendekatan utama yang sering digunakan dan relevan untuk integrasi sistem navigasi cerdas, yaitu \emph{PID controller}, \emph{Braitenberg-based obstacle avoidance}, dan \emph{Pure Pursuit}.

\subsubsection{2.2.9.1 \emph{PID Controller}}

Pengendali PID (\emph{Proportional-Integral-Derivative}) merupakan salah satu algoritma kontrol klasik yang masih luas digunakan karena kesederhanaannya dan efektivitasnya dalam berbagai aplikasi sistem dinamis. PID terdiri dari tiga komponen utama yang bekerja secara sinergis dalam mengurangi kesalahan (\emph{error}) antara nilai keluaran aktual dan nilai referensi sistem. Komponen  \emph{proportional} (\(P\)), memberikan respons sebanding terhadap besarnya \emph{error} saat ini. Semakin besar \emph{error}, semakin besar koreksi yang diberikan, yang membantu sistem mendekati set point secara langsung. Komponen  \emph{integral} (\(I\)), menjumlahkan \emph{error} dari waktu ke waktu untuk mengoreksi kesalahan tetap (\emph{steady-state error}) yang tidak bisa dihilangkan hanya dengan kontrol \emph{proportional}. Komponen \emph{derivative} (\(D\)), bertindak berdasarkan laju perubahan \emph{error}, memberikan peredaman (\emph{damping}) terhadap respons sistem, dan mengurangi gejala \emph{overshoot} serta mempercepat  (\emph{rise time}) menuju \emph{set point}. Secara matematis, sinyal kontrol total \( u(t) \) dalam pengendali PID dapat dirumuskan sebagai:
\begin{equation}
    u(t) = K_p e(t) + K_i \int_{0}^{t} e(\tau) \, d\tau + K_d \frac{d}{dt}e(t),
\end{equation}
di mana \( K_p \), \( K_i \), dan \( K_d \) masing-masing adalah konstanta penguatan (\emph{gain}) untuk komponen \emph{proportional}, \emph{integral}, dan \emph{derivative}, dan \( e(t) \) adalah sinyal kesalahan pada waktu \( t \). Dalam konteks navigasi jalur, pengendali PID umum digunakan untuk mengatur kecepatan linier dan sudut belok robot berdasarkan \emph{cross-track error} (deviasi lateral terhadap jalur) dan \emph{heading error} (perbedaan orientasi). Parameter PID yang dituning dengan baik dapat meminimalkan \emph{steady-state error}, mengontrol \emph{overshoot}, serta mempercepat \emph{rise time} menuju kondisi stabil. Namun, dalam lingkungan yang sangat dinamis atau nonlinear, efektivitas PID dapat menurun karena sifatnya yang tidak adaptif terhadap perubahan konteks \cite{astrom1995pid}.


\subsubsection{2.2.9.2 Braitenberg \emph{Obstacle Avoidance}}

Penghindaran rintangan berdasarkan prinsip \emph{Braitenberg} merupakan pendekatan reaktif yang meniru perilaku biologis sederhana, di mana sensor (misalnya ultrasonik atau LiDAR) dihubungkan langsung ke aktuator secara fungsional. Dalam bentuk dasar, intensitas pembacaan sensor mempengaruhi kecepatan roda secara langsung untuk menghindari objek yang terdeteksi. Meskipun sangat sederhana, pendekatan ini efektif dalam menghindari tabrakan di lingkungan yang tidak terstruktur. Namun, karena sifatnya yang reaktif, metode ini cenderung menghasilkan gerakan tidak halus dan kurang optimal dalam konteks navigasi jalur kompleks. Oleh karena itu, Braitenberg sering digunakan sebagai lapisan reaktif tambahan pada sistem pengendalian berbasis jalur seperti PID atau Pure Pursuit \cite{braitenberg1986vehicles}.

\subsubsection{2.2.9.3 Pure Pursuit untuk Pelacakan Jalur}

\emph{Pure Pursuit} adalah algoritma pelacakan jalur berbasis geometri yang dirancang untuk mengarahkan robot menuju sebuah titik target (\emph{look-ahead point}) yang berada di depan lintasan referensi. Algoritma ini populer karena kesederhanaannya serta kemampuannya menghasilkan lintasan yang halus, terutama pada sistem robotik non-ackermann seperti robot dengan \emph{differential drive}. Prinsip kerja \emph{Pure Pursuit} adalah membentuk lintasan lengkung dari posisi robot menuju titik target. Sudut belok atau kelengkungan lintasan dihitung berdasarkan sudut \( \alpha \), yaitu sudut antara arah heading robot saat ini dan garis lurus ke titik target, serta jarak \emph{look-ahead} \(L_d\). Kelengkungan lintasan ditentukan oleh persamaan:
\begin{equation}
    \kappa = \frac{2 \cdot \sin(\alpha)}{L_d},
\end{equation}
di mana \( \kappa \) kemudian dikonversi menjadi kecepatan sudut:
\begin{equation}
    \omega = v \cdot \kappa,
\end{equation}
dengan \( v \) adalah kecepatan linier robot.

\emph{Look ahed distance} \(L_d\) sangat memengaruhi karakteristik gerakan yang dihasilkan oleh algoritma ini. Ketika nilai \(L_d\) terlalu kecil, robot menjadi sangat responsif terhadap perubahan arah, sehingga sering kali menghasilkan lintasan yang terlalu tajam atau osilatif, terutama pada kecepatan tinggi. Sebaliknya, jika \(L_d\) terlalu besar, robot cenderung mengambil jalur yang lebih lurus dan memotong tikungan, sehingga berpotensi menyimpang dari lintasan referensi. Oleh karena itu, pemilihan nilai \(L_d\) perlu mempertimbangkan kecepatan dan konteks lingkungan. Beberapa implementasi bahkan menggunakan skema adaptif yang menyesuaikan \(L_d\) secara dinamis seiring dengan kecepatan robot untuk menjaga keseimbangan antara stabilitas dan akurasi pelacakan. Metode \emph{Pure Pursuit} sangat cocok untuk navigasi jalur dengan geometri kompleks karena mampu menyesuaikan arah gerak secara kontinu berdasarkan posisi dan heading saat ini. Untuk meningkatkan stabilitas sistem dan mengurangi osilasi gerakan, algoritma ini sering dikombinasikan dengan pengendali PID yang mengatur kecepatan sudut berdasarkan nilai kelengkungan yang dihitung \cite{coulter1992implementation}.


\subsection{\emph{Computer Vision}}

\emph{Computer vision} merupakan cabang dari kecerdasan buatan yang bertujuan untuk memungkinkan komputer memahami dan menganalisis informasi dari citra maupun video digital. Tidak seperti manusia yang secara intuitif dapat mengenali objek, memperkirakan jarak, dan memahami konteks visual, sistem \emph{computer vision} memerlukan algoritma matematis dan pemrograman tingkat lanjut untuk meniru proses tersebut. Salah satu perkembangan penting dalam bidang ini adalah penggunaan arsitektur \emph{Convolutional Neural Network} (CNN), yang menjadi fondasi dalam berbagai aplikasi seperti klasifikasi citra, segmentasi semantik, serta deteksi objek. Salah satu algoritma deteksi objek yang paling populer adalah \emph{You Only Look Once} (YOLO). YOLO telah mengalami banyak peningkatan sejak pertama kali diperkenalkan pada tahun 2016, dengan tujuan untuk meningkatkan kecepatan dan akurasi deteksi objek dalam citra. YOLO bekerja dengan membagi citra menjadi grid dan memprediksi bounding box serta kelas objek dalam satu langkah pemrosesan, menjadikannya sangat efisien untuk aplikasi \emph{real-time}.


\subsubsection{2.2.10.1 YOLOv8}
\emph{YOLOv8} merupakan pengembangan dari algoritma \emph{You Only Look Once} yang menggunakan pendekatan deteksi satu tahap (\emph{one-stage detection}). Model ini mendeteksi objek secara langsung dari keseluruhan citra dalam satu proses, tanpa perlu memisahkannya menjadi beberapa bagian seperti pada metode dua tahap. Peningkatan utama pada \emph{YOLOv8} meliputi optimasi deteksi multi-skala, penggunaan \emph{backbone} yang efisien seperti \emph{CSPDarknet}, serta \emph{neck} yang ditingkatkan untuk ekstraksi fitur, seperti \emph{PANet}. Arsitektur ini memungkinkan deteksi yang cepat dan akurat, serta efisien untuk diterapkan pada perangkat dengan daya komputasi terbatas.  Dengan kombinasi kecepatan, akurasi, dan efisiensi, \emph{YOLOv8} banyak digunakan dalam berbagai aplikasi \emph{computer vision}, termasuk sistem keamanan, kendaraan otonom, dan pengolahan citra medis \cite{yolov8}.


\subsubsection{2.2.10.2 OpenVINO}
OpenVINO (\emph{Open Visual Inference and Neural Network Optimization}) adalah toolkit dari Intel yang dirancang untuk mempercepat inferensi model \emph{deep learning} pada perangkat keras yang berbeda, termasuk CPU, GPU, dan VPU. Dengan menggunakan OpenVINO, model \emph{YOLOv8} dapat dioptimalkan untuk berjalan pada perangkat edge dengan efisiensi tinggi. OpenVINO mendukung berbagai format model, termasuk TensorFlow, PyTorch, dan ONNX, serta menyediakan alat untuk konversi model ke format \emph{Intermediate Representation} (IR) yang lebih efisien. Selain itu, OpenVINO juga menawarkan optimasi presisi data seperti \texttt{FP16} (16-bit floating point) dan \texttt{INT8} (8-bit integer), yang memungkinkan peningkatan performa inferensi dengan kompromi minimal terhadap akurasi. Dengan demikian, OpenVINO sangat cocok untuk aplikasi \emph{real-time} seperti pemantauan suhu termal pada gardu listrik \cite{openvino2021toolkit}.

\subsection{Metode Evaluasi dalam \emph{Computer Vision}}

Evaluasi performa model dalam \emph{computer vision} sangat penting untuk mengukur efektivitas sistem dalam mendeteksi dan mengklasifikasikan objek secara akurat. Beberapa metrik evaluasi yang umum digunakan meliputi \emph{confusion matrix}, \emph{precision}, \emph{recall}, \emph{accuracy}, \emph{F1 score}, dan \emph{loss function}. Masing-masing metrik memberikan perspektif yang berbeda mengenai kinerja model, baik dari segi akurasi keseluruhan, kesalahan klasifikasi, maupun keseimbangan antara kesalahan positif dan negatif.

\subsubsection{2.2.11.1 \emph{Confusion Matrix}}
\emph{Confusion matrix} merupakan sebuah representasi dalam bentuk matriks yang digunakan untuk mengevaluasi kinerja model klasifikasi, khususnya dalam konteks pembelajaran mesin. Matriks ini menunjukkan distribusi hasil prediksi model terhadap data uji, dengan mengklasifikasikan hasil tersebut ke dalam kategori prediksi benar dan salah. Untuk kasus klasifikasi biner, \emph{confusion matrix} terdiri dari empat komponen utama, yaitu \emph{True Positive (TP)}, yang menunjukkan jumlah data positif yang berhasil diklasifikasikan dengan benar; \emph{True Negative (TN)}, yaitu jumlah data negatif yang juga diprediksi dengan tepat; \emph{False Positive (FP)}, yaitu data negatif yang salah diklasifikasikan sebagai positif; serta \emph{False Negative (FN)}, yaitu data positif yang secara keliru diprediksi sebagai negatif.




\begin{figure}[H]
  \centering
  \includegraphics[scale=0.7]{gambar/bab2/cf.png}
  \caption{Contoh \emph{confusion matrix} untuk klasifikasi biner.}
  \label{fig:cf}
  \footnotesize{\textbf{Sumber:} V7Labs (2022)}
\end{figure}

Melalui analisis terhadap keempat komponen ini, pengguna dapat mengevaluasi secara lebih mendalam jenis kesalahan yang terjadi pada model, seperti kecenderungan overfitting terhadap kelas tertentu atau ketidakseimbangan prediksi. Informasi dari \emph{confusion matrix} juga menjadi dasar dalam perhitungan metrik evaluasi lain seperti \emph{accuracy}, \emph{precision}, \emph{recall}, dan \emph{F1-score}, yang lebih menggambarkan performa model secara menyeluruh dalam konteks prediksi klasifikasi.

\subsubsection{A. \emph{Precision}, \emph{Recall}, dan \emph{Accuracy}}

\emph{Precision} menunjukkan seberapa akurat prediksi positif yang dilakukan model, sementara \emph{recall} mengukur seberapa banyak data positif yang berhasil dikenali oleh model. \emph{Accuracy} mengukur persentase keseluruhan prediksi yang benar.

\begin{equation}
  Precision = \frac{TP}{TP + FP}
\end{equation}

\begin{equation}
  Recall = \frac{TP}{TP + FN}
\end{equation}

\begin{equation}
  Accuracy = \frac{TP + TN}{TP + TN + FP + FN}
\end{equation}

\subsubsection{B. \emph{F1 Score}}

\emph{F1 score} adalah rata-rata harmonis antara \emph{precision} dan \emph{recall}. Metrik ini sangat berguna ketika terdapat ketidakseimbangan kelas pada data.

\begin{equation}
  F_1 = 2 \times \frac{Precision \times Recall}{Precision + Recall}
\end{equation}

\subsubsection{C. \emph{Loss Function}}

\emph{Loss function} berfungsi untuk mengukur seberapa jauh prediksi model dari nilai sebenarnya. Fungsi ini digunakan selama pelatihan model untuk meminimalkan kesalahan prediksi. Salah satu fungsi kerugian yang paling umum digunakan dalam klasifikasi adalah \emph{cross-entropy loss}.

\begin{equation}
L = -\sum_{i=1}^{C} y_i \log(p_i)
\end{equation}

Di mana \( C \) adalah jumlah kelas, \( y_i \) adalah label asli untuk kelas ke-\(i\), dan \( p_i \) adalah probabilitas prediksi model terhadap kelas ke-\(i\).


\subsection{\emph{Web GUI}}

\emph{Web Graphical User Interface} atau \emph{Web GUI} merupakan evolusi dari antarmuka pengguna tradisional yang umumnya dijalankan melalui aplikasi desktop. Dengan memanfaatkan teknologi web, \emph{Web GUI} memungkinkan pengguna untuk berinteraksi dengan sistem atau aplikasi melalui peramban web (\emph{browser}) tanpa memerlukan instalasi perangkat lunak tambahan, sehingga memberikan fleksibilitas tinggi dan aksesibilitas yang luas \cite{krug2014dont}. Keunggulan utama dari \emph{Web GUI} dibandingkan dengan GUI konvensional terletak pada sifatnya yang lintas platform dan berbasis cloud. Selama pengguna memiliki koneksi internet dan peramban modern, aplikasi dapat diakses secara langsung tanpa bergantung pada sistem operasi tertentu \cite{murugesan2007web}. Selain itu, proses pemeliharaan dan pembaruan (\emph{update}) menjadi lebih efisien, karena seluruh perubahan dapat dilakukan secara terpusat pada sisi server tanpa perlu mendistribusikan ulang perangkat lunak kepada setiap pengguna. Dalam konteks pengembangan sistem modern, \emph{Web GUI} sangat didukung oleh berbagai teknologi berbasis JavaScript dan framework pendukung yang memungkinkan pengembangan antarmuka yang cepat, responsif, dan \emph{scalable}. 

\subsubsection{2.2.12.1 React.js}

React.js merupakan salah satu \emph{library} \emph{JavaScript} yang umum digunakan dalam pengembangan antarmuka pengguna (\emph{user interface}) yang dinamis dan interaktif. React mengadopsi pendekatan berbasis \emph{component}, yaitu unit antarmuka yang terpisah dan modular, sehingga memudahkan proses pengembangan, pemeliharaan, serta pengujian perangkat lunak. Komponen dalam React dapat dibangun melalui dua pendekatan utama, yaitu \emph{class components} dan \emph{functional components}, yang keduanya bertanggung jawab dalam proses \emph{rendering} elemen antarmuka berdasarkan \emph{props} dan \emph{state} \cite{Panjaitan2021}. Efisiensi dalam pembaruan antarmuka didukung oleh penerapan \emph{Virtual DOM}, yang memungkinkan perubahan hanya terjadi pada bagian yang terdampak tanpa memuat ulang seluruh halaman. Pendekatan ini tidak hanya meningkatkan keterbacaan dan perawatan kode, tetapi juga mendukung prinsip \emph{reusability}, yakni penggunaan kembali komponen di berbagai konteks pengembangan. Dengan karakteristik tersebut, React.js menjadi salah satu teknologi dominan dalam pengembangan aplikasi web berbasis \emph{single-page application} (\emph{SPA}) \cite{Panjaitan2021}.

\subsubsection{2.2.12.2 Next.js}
Next.js merupakan \emph{framework} berbasis \emph{React} yang dikembangkan untuk mendukung pengembangan aplikasi web \emph{full-stack} secara efisien dan terstruktur. Sebagai ekstensi dari \emph{React}, Next.js menyederhanakan berbagai konfigurasi tingkat rendah seperti \emph{module bundling}, \emph{routing}, dan \emph{code splitting}, yang umumnya memerlukan pengaturan manual dalam proyek \emph{React} murni \cite{Nextjs2024}. Kemampuan ini memungkinkan pengembang untuk fokus pada pengembangan fitur tanpa harus terlibat langsung dalam proses pengelolaan infrastruktur aplikasi. Next.js mengintegrasikan secara native pendekatan \emph{server-side rendering} (SSR) dan \emph{static site generation} (SSG), dua teknik rendering yang krusial dalam optimalisasi performa dan aksesibilitas halaman web. SSR memungkinkan halaman dirender di sisi server sebelum dikirim ke klien, sehingga mengurangi waktu muat awal pada web browswer. Di sisi lain, SSG menghasilkan halaman statis pada waktu kompilasi, yang secara signifikan menurunkan beban server dan mempercepat distribusi konten. Kombinasi kedua pendekatan ini memberikan fleksibilitas tinggi dalam strategi rendering berdasarkan kebutuhan spesifik aplikasi. Dalam konteks pengembangan berbasis komponen, integrasi React dan Next.js mendukung pengembangan antarmuka interaktif dengan performa tinggi. Studi mutakhir menunjukkan bahwa implementasi \emph{framework} seperti Next.js dapat menurunkan latensi interaksi, meningkatkan \emph{user engagement}, serta mendukung skalabilitas dalam arsitektur perangkat lunak modern \cite{Nextjs2024}.

\subsubsection{2.2.12.3 Tailwind CSS}

Tailwind CSS merupakan \emph{framework} \emph{utility-first} berbasis CSS yang mengedepankan penggunaan kelas utilitas langsung dalam markup untuk membentuk antarmuka pengguna yang responsif dan terstruktur. Berbeda dengan \emph{framework} tradisional yang menyediakan komponen siap pakai, pendekatan Tailwind memungkinkan fleksibilitas tinggi dan iterasi cepat dalam pengembangan desain kustom \cite{Azhariyah2024}. Integrasinya yang seamless dengan \emph{framework} JavaScript seperti React dan Angular memperkuat efisiensi pengembangan berbasis komponen, sekaligus mengurangi kebutuhan akan file CSS yang besar. Dukungan bawaan terhadap desain responsif menjadikan Tailwind CSS relevan untuk pengembangan aplikasi lintas perangkat dengan konsistensi visual yang tinggi.

\subsubsection{2.2.12.4 ShadCN}

ShadCN UI merupakan kumpulan komponen \emph{React} modern yang bersifat \emph{open-source}, namun tidak dikemas sebagai pustaka tradisional yang dapat diinstal melalui \emph{Node Package Manager} (\emph{npm}). Sebaliknya, pengembang menyalin langsung kode sumber komponen ke dalam basis kode proyek, sehingga memberikan fleksibilitas tinggi dalam modifikasi fungsionalitas dan penataan. Komponen-komponen ini menggunakan \emph{Tailwind CSS} untuk styling, dengan desain awal yang konsisten dan minimalis. ShadCN menyediakan beragam komponen antarmuka umum seperti \emph{dialog}, \emph{input}, \emph{checkbox}, dan \emph{table}, serta mendukung integrasi cepat melalui perintah \emph{CLI} untuk menyalin kode ke dalam direktori lokal proyek \cite{Shadcn2024}. Pendekatan ini menghasilkan aplikasi yang lebih ringan dan modular, karena hanya menyertakan komponen yang benar-benar dibutuhkan.

\subsubsection{2.2.12.4 Express.js}
Express.js adalah \emph{framework} minimalis untuk Node.js yang dirancang untuk membangun aplikasi web dan API dengan cepat dan efisien. Dengan arsitektur yang sederhana dan fleksibel, Express memungkinkan pengembang untuk membuat aplikasi dengan struktur yang jelas dan mudah dipelihara. Salah satu fitur utama dari Express adalah kemampuannya untuk menangani berbagai jenis permintaan HTTP, serta mendukung middleware yang memungkinkan penanganan permintaan secara modular. Middleware ini dapat digunakan untuk berbagai tujuan, seperti otentikasi, pengolahan data, dan penanganan kesalahan \cite{express2023docs}. Selain itu, Express juga memiliki ekosistem yang luas dengan banyak pustaka tambahan yang dapat diintegrasikan untuk memperluas fungsionalitas aplikasi. Dengan demikian, Express.js menjadi pilihan populer bagi pengembang yang ingin membangun aplikasi web modern dengan cepat dan efisien.

\subsubsection{2.2.12.5 PostgreSQL}
PostgreSQL, sebuah sistem manajemen basis data relasional (RDBMS) \emph{open-source} yang terkenal, telah menjadi pilihan utama dalam pengembangan aplikasi berbasis data. Dengan arsitektur yang kuat dan dukungan untuk berbagai fitur, PostgreSQL menawarkan kemampuan yang sangat baik dalam hal penyimpanan, pengambilan, dan manipulasi data. Sistem ini dikenal karena kepatuhannya terhadap standar SQL, serta kemampuannya untuk menangani transaksi kompleks dengan tingkat konsistensi yang tinggi \cite{postgresql2023docs}. PostgreSQL juga mendukung berbagai tipe data, termasuk JSON dan XML, yang memungkinkan fleksibilitas dalam menyimpan data semi-terstruktur dan tidak terstruktur. Selain itu, sistem ini memiliki ekosistem yang kaya dengan banyak ekstensi dan modul tambahan yang dapat meningkatkan fungsionalitasnya lebih lanjut \cite{postgresql2023docs}.


\subsubsection{2.2.12.6 WebSocket}
WebSocket adalah protokol komunikasi dua arah yang beroperasi di atas koneksi TCP yang persisten, memungkinkan pertukaran pesan antara klien dan server secara efisien. Protokol ini dirancang untuk mengatasi kekurangan teknologi komunikasi dua arah yang ada sebelumnya, yang sering menggunakan \emph{HTTP} sebagai lapisan transportasi. Dengan memanfaatkan mekanisme \emph{upgrade} dari \emph{HTTP}, WebSocket dapat membuka saluran komunikasi bidirectional yang lebih efisien dibandingkan dengan pendekatan tradisional seperti \emph{polling} atau \emph{long polling} \cite{Fette2011}. Keunggulan utama dari WebSocket terletak pada kemampuannya untuk mengurangi overhead komunikasi, yang sangat penting dalam aplikasi yang memerlukan pembaruan real-time, seperti aplikasi \emph{chat}, permainan daring, dan sistem \emph{IoT}. Penelitian menunjukkan bahwa WebSocket tidak hanya meningkatkan efisiensi komunikasi, tetapi juga memungkinkan pengembangan aplikasi kolaboratif yang lebih responsif dan interaktif \cite{Milsap2019}. Dengan demikian, WebSocket menjadi solusi yang ideal untuk aplikasi yang membutuhkan latensi rendah dan kecepatan tinggi dalam pertukaran data.


